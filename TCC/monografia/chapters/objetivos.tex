
\section{Objetivo}

A taxa de aumento extra de espaço causado pela geração de parte redundante é representado por \textit{overhead} de espaço.
Por exemplo, na replicação de arquivo que possui apenas uma réplica, o \textit{overhead} é de 100\%, pois cada um arquivo será criada uma parte redundante.
O nosso objetivo neste trabalho é a construção de um sistema de arquivos distribuídos tolerante a falha de forma que apresenta menor \textit{overhead} de espaço que o método da replicação de arquivo.
\\

Para isso, nós observamos em conceitos da tecnologia RAID, pois notamos alguns pontos semelhantes com o sistema de arquivos distribuídos, como o armazenamento de forma distribuído e a preocupação com segurança dos arquivos.
Além disso, a utilização da replicação de arquivo e do outro método diferente para atribuição da redundância também foi motivo para colocar atenção nesta tecnologia.
A outra forma de criar as partes redundantes é a geração de paridades, que são calculadas baseando em blocos divididos a partir do próprio arquivo.
Dessa forma, é possível de reduzir o \textit{overhead} de espaço até 50\%, a metade da replicação com uma cópia, quando as paridades forem geradas de forma mais simples, ou até menos, dependendo da condição utilizada para calcular a paridade.
\\

Também notamos no fato de que a tecnologia RAID foi inicialmente desenvolvido para aumentar a taxa de transferência da unidade de armazenamento.
A motivação inicial desta invenção era reusar os discos rígidos que não estavam mais em uso, por serem substituídos pelo outros mais novos com melhor performance.
O problema para fazer o reuso dos dispositivos era a baixa velocidade de leitura/escrita que estes apresentavam, por serem aparelhos relativamente velhos. 
Para cobrir esta baixa velocidade de transferência, os discos foram colocados para trabalhar em conjunto, ou seja, fazer um acesso simultâneo em vários discos para aumentar a vazão total de dados transferidos, resultando em aumento da taxa de transferência .
Assim, a introdução de conceito do RAID, não trata apenas de questão da segurança de arquivos, mas também no desempenho do sistema.  
\\
%Como uma consequência extra de introdução do RAID, é esperado que tenha a melhoria na distribuição da carga de operações entre os servidores, pois os arquivos estão armazenados espalhadamente sobre todos os nós do sistema, levando na redução da probabilidade de ocorrer a concentração de acesso em um dos nós.

Para que o armazenamento de arquivo seja administrado de forma eficiente, contribuindo para aumento de desempenho do sistema, adotamos uma abordagem que separa o arquivo em duas partes para distribuir a carga de gerenciamento. 
Uma parte é o metadado, a que guarda as informações do arquivo, como nome, tamanho ou localização. E a outra parte é o dado, a que armazena o próprio conteúdo deste arquivo. 
Desta forma, o sistema será constituído por dois tipos diferentes de conjunto dos servidores, um que administra os metadados dos arquivos e outro que armazena os dados dos arquivos. 
\\

Da mesma forma que a confiabilidade do conteúdo dos arquivos é assegurado por conceito de RAID, os metadados armazenados também tem que ser protegido.
Porém, não podemos usar o RAID para fazer isso, pois nos servidores de metadados não acontecem simples entrada/saída de dados, mas também fornecem um serviço para indicar localização de conteúdo dos arquivos. 
Por isso, para garantir a segurança dos metadados, junto com o serviço, utilizamos o \textit{BFT-SMaRt}, uma biblioteca de Java que disponibiliza as ferramentas para desenvolver um serviço tolerante a falha, baseado na técnica de replicação de serviço.
\\


%e da mesma forma os metadados serão protegidos através de BFT-SMaRt, uma biblioteca JAVA que fornece a tolerância a falha no serviço, realizada com uso de método de replicação do serviço.\\


Assim, o nosso trabalho está focalizado em construção de um sistema de arquivos distribuídos utilizando os conceitos de RAID.
Será implementado algumas configurações de RAID, incluindo uma que utiliza a replicação de arquivo e outra que utiliza a geração de paridade, e posteriormente será feito a comparação entre as configurações desenvolvidas no sistema, para verificar a diferença de desempenho elas.
\\

%Também será feito a comparação de desempenho entre os dois sistemas, um que usa método de replicação e outro que usa método de RAID.
%Comparando os dois métodos de redundância, será possível a visualização das vantagens e desvantagens que existem entre eles.










