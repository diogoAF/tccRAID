
\section{Objetivo}

O nosso objeivo neste artigo é a construção de um sistema de arquivos distribuído baseado na computação em nuvem que possui menor \textit{overhead} de espaço, comparado a 200\% de método de replicação, sem degradar muito o performance e a confiabilidade na armazenamento de dados. \\

Para isso, nós observamos em técnica de redundância usado na tecnologia RAID, onde tem possibilidade de diminuir o espaço ocupado pela parte redundante com atribuição de paridade nos arquivos, ao invés de simples replicação. Dessa forma, é possível de reduzir o \textit{overhead} de espaço até 50\%, quando for gerada uma paridade para cada duas partes de arquivo original, ou até menos, dependendo da configuração utilizada para calcular as paridades.
Além de otimização de uso de espaço para armazenamento, também é esperado a melhoria na distribuição de caga entre os servidores. Como os arquivos estão fragmentados em blocos e espalhados em todos os servidores, a probabilidade de ocorrer a concentração de acesso em um servidor tende a ser reduzida.
Também notamos no fato de que a tecnologia RAID foi inicialmente desenvolvido para aumentar a taxa de transferência da unidade de armazenamento, mantendo a mesma velocidade de leitura/estrita. Isso foi realizado com leitura/escrita simultânea em vários discos rígidos, aumentando o tamanho total de dados transferidos por vez numa operação de entrada/saída. Esta ideia também foi incorporado no nosso sistema, uma vez que no sistema distribuído é requisito básico o uso de vários servidores para trabalhar em conjunto.\\

Para gerenciar o armazenamento de forma eficientemente, todos os arquivos são divididos em duas partes, o metadado, as informações sobre arquivo, e o conteúdo, o dado em si do arquivo. 
Assim, no conjunto de servidor vai ter dois tipos diferentes de nós, alguns que armazena os metadados dos arquivos e outros que armazena o conteúdo dos arquivos.
A confiabilidade do conteúdo do arquivo é assegurado com conceito de RAID, e da mesma forma os metadados serão protegidos através de BFT-SMaRt, uma biblioteca JAVA que fornece a tolerância a falha, realizada com uso de método de replicação do serviço.\\

%Assim, a nossa abordagem está focalizado em construção de uma forma eficiente para armazenar os arquivos com segurança no sistema de arquivos distribuído.

Também será feito a comparação de desempenho entre os dois sistemas, um que usa método de replicação e outro que usa método de RAID.
Comparando os dois métodos de redundância, será possível a visualização das vantagens e desvantagens que existem entre eles.










