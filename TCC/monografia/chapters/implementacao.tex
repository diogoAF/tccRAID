
\section{Implementação}
Nesta seção serão apresentados os detalhes sobre a implementação do sistema, incluindo informações sobre a programação e algoritmos utilizados. Todo o \textit{software} foi desenvolvido na linguagem Java com auxilio de várias bibliotecas padrões e do \textit{BFT-SMaRt}, o qual foi apresentado com mais detalhes no Capítulo 4. 
\\


\subsection{Árvode de Diretórios}
Nesta seção será apresentado o pacote responsável por gerenciar todas as estruturas de diretório do sistema, junto com as classes contidas nele.
\\

\begin{itemize}
	\item \textbf{dt}
	\begin{itemize}
		\item DirectoryNode;
		\item DirectoryTree;
		\item LockList;
		\item LockType;
		\item Metadata.
	\end{itemize}
	\item \textbf{dt.directory}
	\begin{itemize}
		\item DirEntries;
		\item Directory.
	\end{itemize}
	\item \textbf{dt.file}
	\begin{itemize}
		\item Block;
		\item BlockInfo;
		\item BlockInfoList;
		\item FileDFS.
	\end{itemize}
\end{itemize}

As classes contidas no pacote \textit{dt} lidam com todas as características e operações refentes aos diretórios. \textbf{LockType} e \textbf{LockList} são utilizadas para gerenciar o controle de acesso aos diretórios. Sendo a primeira responsável por enumerar os tipos dos estados de acesso dos diretórios, enquanto a segunda gerência uma lista, na forma de \textit{ArrayList}, dos arquivos ou diretório que estão abertos pelos usuários. A função desta lista é listar os programas clientes que requisitaram ao serviço de metadados a atualização do estado de acesso. 
\\

A classe \textbf{Metadata} contêm e manipula as informações de metadado dos arquivos. Como é mostrado no trecho de Código~\ref{code:metadata}, a classe \textbf{Metadata} implementa a interface \textit{Serializable} para que possa transformar o conteúdo de seus campos em sequências de \textit{bytes} para posteriormente serem enviados através da rede.
\\

\begin{lstlisting}[basicstyle=\ttfamily\footnotesize, frame=single, caption=Classe Metadata, label=code:metadata]	
public class Metadata implements Serializable {
	private long creationTimeL;
	private long lastAccessTimeL;
	private long lastModifiedTimeL;
	
	private long size;
	
	private int lock;
	
			...
			
}
\end{lstlisting}	

Por fim, \textbf{DirectoryNode} e \textbf{DirectoryTree} compõem a árvore de diretórios do serviço de metadados. A primeira refere-se a cada nó da árvore e a segunda sobre a árvore em si, carregando o diretório raiz e os métodos das operações relacionadas.
\\

\begin{lstlisting}[basicstyle=\ttfamily\footnotesize, frame=single, caption=Declaração e os campos da classe Directory, label=code:directory]
public class Directory extends DirectoryNode {
	private HashMap<String, Directory> dirs;
	private HashMap<String, FileDFS>   files;
	
	...
}
\end{lstlisting}

Dentro do pacote \textit{dt}, existem dois outros pacotes. O \textit{directory} é responsável por tratar os diretórios em si, composto pelas classes \textbf{Directory} e \textbf{DirEntries}.
A classe \textit{Directory} trata dos diretórios em si, composta por vários métodos de operações sobre pastas além de possuir dois campos \textit{HashMaps}. Um para armazenar as informações das subpastas e outro para as informações dos arquivos contidos no diretório referenciado. Por se tratar de um dos tipos de nó da árvore de diretórios, o \textbf{Directory} herda da classe \textit{DirectoryNode}, o trecho de Código~\ref{code:directory} exemplifica essa característica.
\\

A classe \textit{DirEntries} é usada pelo serviço de metadados para informar aos clientes os registros do diretório aberto pelo usuário. Os registros são apresentados na forma de uma lista composta pelos nomes dos subdiretórios e arquivos contidos no diretório. 
\\

O segundo pacote dentro de \textit{dt} é nomeado de \textit{file}, o qual compõem as classes responsáveis pelo gerenciamento das informações sobre arquivos ou dos dados armazenados na forma de blocos. 
O \textbf{FileDFS} é a classe principal deste pacote, sendo encarregando pela parte das informações sobre os arquivos. Apesar disso, o \textbf{FileDFS} não possui muitas funcionalidades além de carregar a classe \textit{BlockInfoList}, como exemplificado no trecho de Código~\ref{code:filedfs}. Note que por ser tratar de um tipo de nó da árvore de diretórios, o \textbf{FileDFS} herda as características da classe \textit{DirectoryNode}.
\\

\begin{lstlisting}[basicstyle=\ttfamily\footnotesize, frame=single, caption=Classe FileDFS, label=code:filedfs]
public class FileDFS extends DirectoryNode {
	private BlockInfoList blockList;
	
	public FileDFS(String name, Directory parent, Metadata metadata, BlockInfoList blockList) {
		super(name, parent, metadata);	
		this.blockList = blockList;
	}
	
	public BlockInfoList getBlockList() {
		return blockList;
	}
	
}
\end{lstlisting}

Como demonstrado no trecho de Código~\ref{code:blockinfolist}, o \textbf{BlockInfoList} é a classe que armazenas as propriedades do arquivo: a lista de \textit{BlockInfo}, o tamanho dos blocos, o tipo de RAID utilizado e o número dos servidores que armazenam os blocos.
A sua função principal é permitir ao serviço de metadados informar aos clientes os dados operativos de qualquer arquivo. Tal função é necessário quando precisa-se enviar ou receber os blocos de dados de algum arquivo. Cada \textbf{BlockInfo} gerencia a localização de um bloco, indicando o endereço do servidor onde está armazenado e o identificador do bloco referente.
\\

\begin{lstlisting}[basicstyle=\ttfamily\footnotesize, frame=single, caption=Declaração e os campos da classe BlockInfoList, label=code:blockinfolist]
public class BlockInfoList implements Serializable {
	private ArrayList<BlockInfo> blocks;
	private long blockSize;
	private int  raidType;
	private int  nServers;
 
			...
}
\end{lstlisting}

A classe \textbf{Block} é encarregada de tratar o conteúdo dos arquivos, gerenciando em formato de blocos.
Ela é responsável pela transferência de dados dos arquivos, contidos em blocos, entre cliente e o serviço de armazenamento. Além de possuir a atribuição de informar os servidores o identificador de cada bloco.
\\

%\textbf{Block, BlockInfo e BlockInfoList} referentes aos blocos de dados nos quais os arquivos devem ser divididos antes de serem enviados pelo cliente ao serviço de armazenamento. 

%Dentre as classes previamente sitadas,\textit{Block} pode ser considerada como a principal, visto que é nela onde ficam armazenados os \textit{bytes} de dados, o identificador e os métodos necessários para recuperar os dados ou quebrar um dado arquivo em blocos. 

%Enquanto que \textit{BlockInfo} funciona como a ligação do bloco ao servidor de armazenamento par ao qual ele deve ser enviado, contendo as informações necessárias para que o cliente saiba para quem o bloco deve ser enviado. Por fim, \textit{BlockInfoList} possui os meios para gerenciar um \textit{ArrayList} dos \textit{BlockInfo} de um arquivo, além de manter o tamanho de cada bloco da lista, o número de servidores de armazenamento e o tipo de RAID em que os blocos foram criados.


\subsection{Serviço de Metadados}
Nesta seção a implementação do Serviço de Metadados é detalhada. O serviço supracitado foi desenvolvido em quatro classes principais e uma auxiliar, as quais são listadas a seguir de acordo com  a estrutura do pacote.
\\

\begin{itemize}
	
	\item server
	\begin{itemize}
		\item ServerList;
	\end{itemize}
	\item server.meta
	\begin{itemize}
		\item RaidType;
		\item ServerConsole;
		\item ServerInfo;
		\item ServerMeta.
	\end{itemize}

\end{itemize}

A classe \textbf{RaidType} é apenas uma enumeração de números inteiros representado os níveis de RAID suportados pelo sistema.
\\

\begin{lstlisting}[basicstyle=\ttfamily\footnotesize, frame=single, caption=Classe ServerConsole, label=code:meta_con]	
public class ServerConsole {
	
	public static void main(String[] args){
		if(args.length < 3) {
			System.out.println("Use: java ServerConsole <processId> <raidType> <nServers> <verbose>");
			System.exit(-1);
		}
		boolean verbose = false;
		boolean test    = true;
		if(args.length > 3) {
			if(args[3].contains("v"))
			verbose = true;
			if(args[3].contains("n"))
			test = false;
		}
		
		new ServerMeta(Integer.parseInt(args[0]), Integer.parseInt(args[1]), Integer.parseInt(args[2]), verbose, test);
	}
}
\end{lstlisting}	

A classe \textbf{ServerConsole} serve apenas como uma interface de inicialização do serviço de metadados, no qual o \textit{ServerConsole} verifica os parâmetros de inicialização, caso tenha algo errado é apresentado uma mensagem de erro e o programa finalizado. Em contra partida, se tudo estiver correto com os parâmetros, a classe \textit{ServerMeta} é instanciada. Como demonstrado no trecho de Código~\ref{code:meta_con}.
\\

A classe \textbf{ServerList} implementa a lista de servidores de armazenamento utilizando a classe \textit{ArrayList} do Java. Cada item da lista é composto por um objeto da classe \textbf{ServerItem}, responsáveis pelas informações sobre servidores de armazenamento, como mostrado no trecho de Código~\ref{code:serv_info}.
\\

\begin{lstlisting}[basicstyle=\ttfamily\footnotesize, frame=single, caption=Campos da classe ServerInfo, label=code:serv_info]	
public class ServerInfo {
	private String hostName;
	private int    port;
	private long   capacity;
	private long   size;
	private long   lastID;
	private long   lastAccessTime;

			...
}
\end{lstlisting}	

A classe \textbf{ServerMeta} é o núcleo do serviço de metadados, possuindo como superclasse o \textit{DefaultSingleRecoverable}, uma classe que pertence a biblioteca \textit{BFT-SMaRt}. Dentro de seu construtor, a classe \textbf{ServerMeta} inicializa a árvore de diretórios e a lista de servidores de armazenamento, como mostrado no trecho de Código~\ref{code:serv_meta_con}.
\\

\begin{lstlisting}[basicstyle=\ttfamily\footnotesize, frame=single, caption=Declaração e construtor da classe ServerMeta, label=code:serv_meta_con]

public class ServerMeta extends DefaultSingleRecoverable {
			...
		
	public ServerMeta(int id){
		new ServiceReplica(id, this, this);
		dt   = new DirectoryTree();
		list = new ServerList(); 
	}

			...
}
\end{lstlisting}	

O método \textit{appExecuteOrdered}(comando), sobrescrito da superclasse e mostrado no trecho de Código~\ref{code:serv_meta_app}, é invocado para atender às requisições por operação onde a ordem de execução importa. O tipo de operação que deve ser executado é passado como parâmetro para a variável \textit{reqType} e retransmitida para o método correspondente, para que ele possa continuar com o processo de execução do comando solicitado. 
\\

Enquanto que as operações que podem ser executadas em qualquer ordem devem ser atendidas pelo método \textit{appExecuteUnordered}(comando), o qual também recebe o comando de execução como parâmetro do método.
\\

\begin{lstlisting}[basicstyle=\ttfamily\footnotesize, frame=single, caption=Métodos para atender às requisições, label=code:serv_meta_app ]		
	@Override
	public byte[] appExecuteOrdered(byte[] command, MessageContext msgCtx) {
				
			...
				
		byte[] resultBytes = null;
		
		try {
			ByteArrayInputStream in  = new ByteArrayInputStream(command);
			ObjectInputStream    ois = new ObjectInputStream(in);
			
			int reqType = ois.readInt();
			
			switch(reqType) {
				case RequestType.CREATEDIR:
				resultBytes = criateDir(ois);
				break;
				
				case RequestType.DELETEDIR:
				resultBytes = deleteDir(ois);
				break;
	
			...
				
		} catch (ClassNotFoundException | IOException e) {
	
			...
			
	}
			
	@Override
	public byte[] executeUnordered(byte[] command, MessageContext msgCtx) {	
	
			...
			
	}
\end{lstlisting}	


Todos os métodos da classe \textit{ServerMeta} possuem um fluxo de execução padronizado da seguinte forma:;  (1) Uma requisição é recebida (2) Executa a requisição solicitada (3) Retorna o resultado para o solicitador da requisição. 
\\

O método \textit{open} mostrado no trecho de Código~\ref{code:serv_meta_open} apresenta o fluxo padrão supracitado para a operação de abrir um arquivo.
\\
\\

\begin{lstlisting}[basicstyle=\ttfamily\footnotesize, frame=single, caption=Exemplo de método da classe ServerMeta, label=code:serv_meta_open]		
	private byte[] open(ObjectInputStream ois) throws ClassNotFoundException, IOException {
		String   currPath = (String)ois.readObject();
		String   tgtName  = (String)ois.readObject();
		long     accTime  = ois.readLong();
		
				...
		
		Directory currDir   = dt.openDirectory(currPath, accTime);
		int       result    = -1;
		long      fileSize  = 0;
		FileDFS   target    = null;
		BlockInfoList bList = null;
		
		if(currDir == null) {
			currDir = dt.getRoot();
			result  = ResultType.FAILURE;
		} else if(!currDir.existFile(tgtName)) {
			result = ResultType.FILENOTEXISTS;
		} else {
			target = currDir.getFile(tgtName);
			if( ( target.isLokedW() ) &&
				( System.currentTimeMillis()-target.getLastAccTime() ) <= 30*1000 ) 
			{
				result = ResultType.FILELOCKED;
			} else {
				target.lockR();
				bList  = target.getBlockList();
				fileSize = target.getMetadata().size();
				result = ResultType.SUCCESS;
			}
		}
	
		ByteArrayOutputStream out = new ByteArrayOutputStream();
		ObjectOutputStream    oos = new ObjectOutputStream(out);
	
		oos.writeInt(result);
		oos.writeObject(currDir.getDirEntries());
		oos.writeObject(bList);
		oos.writeLong(fileSize);
		oos.flush();
		
				...
					
		return out.toByteArray();
	}
\end{lstlisting}


\subsection{Serviço de Armazenamento}
Nesta seção a implementação do Serviço de Armazenamento é detalhada. O serviço supracitado foi desenvolvido em quatro classes, as quais são listadas a seguir.
\\

\begin{itemize}
	\item server.data;
	\begin{itemize}
	\item MetadataModule;
	\item ServerConsole;
	\item Operation;
	\item ServerData.
	\end{itemize}
\end{itemize}

A classe \textbf{ServerConsole} serve apenas como uma interface de inicialização do serviço de armazenamento, no qual o \textit{ServerConsole} verifica os parâmetros de inicialização, caso tenha algo errado é apresentado uma mensagem de erro e o programa finalizado. Em contra partida, se tudo estiver correto com os parâmetros, a classe \textit{ServerData} é instanciada.
\\

A classe \textbf{Operation} é a classe responsável por efetivamente executar as operações solicitadas ao serviço de armazenamento sobre os blocos de arquivos. Ela herda a classe \textit{Thread} para realizar o processamento em \textit{multithread}, para que a classe seja capaz de atender multiplos clientes simultaneamente. Na parte de execução mostrado no trecho de Código~\ref{code:operation} é realizada a chamada do método adequado de acordo com tipo de operação solicitada.
\\

\begin{lstlisting}[basicstyle=\ttfamily\footnotesize, frame=single, caption=Declaração e o método de execução da classe Operation, label=code:operation]		
	public class Operation extends Thread {
		
				...
				
		public void run() {
			if(verbose)
				System.out.println("Cliente conectado do IP "
				+clientSocket.getInetAddress().getHostAddress());
			
			try {
				InputStream in = clientSocket.getInputStream();
				ObjectInputStream ois = new ObjectInputStream(in);
				
				int   reqType = ois.readInt();
				Block block   = (Block)ois.readObject();
				
				switch(reqType) {
					case(RequestType.CREATE):
						create(block);
						break;
				
					case(RequestType.DELETE):
				
				...
				
		}
		
		...
		
	}
\end{lstlisting}

\textbf{ServerData} é a classe responsável por lidar diretamente com os clientes, de modo que é ela quem sabe a porta onde o servidor deve aguardar pela conexão do cliente, a sua capacidade total de armazenamento e quanto do espaço já foi ocupado.
\\

O trecho do código mostrado no Código~\ref{code:serv_data} é a parte principal do fluxo de execução da classe \textit{ServerData}. O fluxo começa com a classe esperando a conexão do cliente, quando a conexão é realizada o fluxo segue para a criação da instancia da classe \textit{Operation}.
\\

\begin{lstlisting}[basicstyle=\ttfamily\footnotesize, frame=single, caption=Declaração e o método de execução da classe Operation, label=code:serv_data]		
	while(true) {
		iterations++;
		if(verbose)
			System.out.println("Aguardando cliente...");
		try {
			Socket clientSocket = serverSocket.accept();
			
			Operation op = new Operation(clientSocket, dirName, verbose);
			op.start();
		} catch (SocketException e) {
			e.printStackTrace();
			System.exit(0);
		}
	}
\end{lstlisting}

A classe \textbf{MetaDataModule} realiza a comunicação com os servidores de metadados. Tal comunicação é alcançada utilizando as facilidades da biblioteca \textit{BFT-SMaRt}, na qual o \textit{MetaDataModule} age no papel de um cliente do protocolo de comunicação solicitando serviços aos servidores de metadados.
\\

\subsection{Cliente}
Esta seção detalhada a implementação do Cliente. O serviço supracitado foi desenvolvido em cinco classes contidos no pacote \textit{client}, as quais são listadas a seguir.
\\

\begin{itemize}
	\item client
	\begin{itemize}
	\item ClientConsole;
	\item ClientDFS;
	\item ClientServerSocket;
	\item Option;
	\item ClientTest.
	\end{itemize}
\end{itemize}

Observe que a classe \textbf{ClientTest} é utilizada apenas para realização de testes, na execução normal ela é ignorada pelo sistema e não deve ser chamada. O trecho de Código~\ref{code:clent_test1} demonstra a inicialização das \textit{threads} da classe \textit{ClientDFS}, responsáveis por efetivamente inicializar os testes.
\\

\begin{lstlisting}[basicstyle=\ttfamily\footnotesize, frame=single, caption=Preparação de teste na classe ClientTest, label=code:clent_test1]		
	long[] values = new long[numThreads];
	Client[] c = new Client[numThreads];
	
	for (int i = 0; i < numThreads; i++) {
		try {
			Thread.sleep(100);
		} catch (InterruptedException e) {
			e.printStackTrace();
		}
	
		System.out.println("Launching client " + (initId + i));
		c[i] = new ClientTest.Client(values, i, opsType, numberOfOps, interval);
	}
		
			...
\end{lstlisting}

Ao inicializar a classe \textbf{ClientTest} é necessário informar, como um dos parâmetros de entrada, qual operação os clientes devem solicitar. O trecho de Código~\ref{code:clent_test2} representa o modo como a operação solicitada é repassada para o restante do sistema. Ao fim da execução de todas as \textit{threads}, apenas a instância portadora do atributo ID inicial imprime os resultados para o usuário antes de finalizar sua execução.
\\

\begin{lstlisting}[basicstyle=\ttfamily\footnotesize, frame=single, caption=Execução de teste na classe ClientTest, label=code:clent_test2]
	for (int i = 0; i < numberOfOps; i++, req++) {
		System.out.print("Sending req " + req + "...");
		
		try {
			switch(opsType) {
				case(READ):
					last_send_instant = System.nanoTime();
					cdfs.open("test_"+id+"_"+i);
					st.store(System.nanoTime() - last_send_instant);
					break;
				
				case(WRITE):
					last_send_instant = System.nanoTime();
					cdfs.create("test", "test_"+id+"_"+i);
					st.store(System.nanoTime() - last_send_instant);
					break;
				
			...
				
		if (id == initId) {
			System.out.println(this.id + " // Average time for " + numberOfOps + " executions (-10%) = " + st.getAverage(true) / 1000 + " us ");
			
			...
				
		}	
			
			...
\end{lstlisting}
 
A classe \textbf{ClientConsole} serve apenas como uma interface entre o usuário e o \textit{software}, na qual o \textit{ClientConsole} apresenta um menu baseado em linha de comando, onde o usuário informa qual operação deseja executar. O \textit{ClientConsole} valida os dados fornecidos e avisa ao módulo responsável por executar a operação desejada.
\\

No trecho de Código~\ref{code:clent_con} é mostrado o método que executa a operação de criação de arquivo. O código demonstra o fluxo que praticamente todos os métodos que executam alguma operação possuem; interação com usuário, chamada do método da classe \textit{ClientDFS}, processamento e resultado.
\\

\begin{lstlisting}[basicstyle=\ttfamily\footnotesize, frame=single, caption=Exemplo de método da classe ClientConsole, label=code:clent_con]
		private void create(Console con) throws ClassNotFoundException, IOException  {
			System.out.println();
			System.out.println("Criar arquivo");
			String srcName  = con.readLine("Nome ou local do arquivo:\n>");
			
			if(srcName.isEmpty())
				return;
			
			int result = c.create(srcName, null);
			
			if(result == ResultType.SUCCESS)
				System.out.println("Arquivo criado");
			else
				reportError(result);
		}
\end{lstlisting}

A classe \textbf{Option} é apenas uma enumeração contendo os tipos de operações suportadas pelo sistema, as quais foram apresentadas e discutidas previamente nesse trabalho.
\\

\textbf{ClientServerSocket} possibilita e gerencia as conexões \textit{Socket} realizadas entre o cliente e os servidores de dados, tais conexões são necessárias para a correta execução de operações entre o cliente e os servidores de armazenamento. As conexões entre o cliente e serviço de metadados ocorrem graças as facilidades providas pela biblioteca \textit{BFT-SMaRt}. O método \textit{open} mostrado no trecho de Código~\ref{code:clent_socket1} é invocado para receber os blocos de dados, enviados pelo serviço de armazenamento.
\\

O método \textit{send} é privado para que pudesse ser chamado por métodos públicos,  tais como os métodos \textit{create} e \textit{delete}. O primeiro utiliza o método \textit{send} para realizar o envio dos blocos de arquivo. Enquanto o segundo o utiliza para enviar uma requisição de deletar um bloco. O trecho de Código~\ref{code:clent_socket2} exemplifica essas relação entre os três métodos citados.
\\
\\
\\
\\
\\

\begin{lstlisting}[basicstyle=\ttfamily\footnotesize, frame=single, caption=Exemplo de método da classe ClientServerSocket, label=code:clent_socket1]	
	public byte[] open(Block block) throws ConnectException  {
		int triedCount = 0;
		
		while(true) {
			try {
				clientSocket = new Socket(blockInfo.getHostName(), blockInfo.getPort());
				System.out.println("O cliente se conectou ao servidor na porta " 
				+ blockInfo.getPort());
				
				ByteArrayOutputStream bos = new ByteArrayOutputStream();
				ObjectOutputStream    oos = new ObjectOutputStream(bos);
				
				oos.writeInt(RequestType.OPEN);
				oos.writeObject(block);
				oos.flush();
				
				OutputStream out = clientSocket.getOutputStream();
				
				out.write(bos.toByteArray());
			
				InputStream         is = clientSocket.getInputStream();
				BufferedInputStream in = new BufferedInputStream(is);
				
				while(is.available() == 0);
				
				byte[] buffer = new byte[BUFFER_SIZE];
				int length = in.read(buffer);
				System.out.println(length);
			
				return Arrays.copyOfRange(buffer, 0, length);
			} catch(ConnectException | UnknownHostException e) {
				try {
					Thread.sleep(10*1000);
				} catch (InterruptedException e1) {
					e1.printStackTrace();
					System.exit(-1);
				}
				
				triedCount++;
				if(triedCount>3) {
					System.out.println("O cliente nao conseguiu conectar no servidor");
					throw new ConnectException();
				}
			} catch (IOException e) {
				...
			}
		}
	}
\end{lstlisting}

A classe \textbf{ClientDFS} pode ser considerada como a classe principal do lado cliente do sistema. Visto que é nela onde estão concentrados os métodos para realização de todas as operações listadas na enumeração da classe \textit{Option}.
\\

\begin{lstlisting}[basicstyle=\ttfamily\footnotesize, frame=single, caption=Exemplo de métodos da classe ClientServerSocket, label=code:clent_socket2]	
	public void create(Block block) throws ConnectException  {
		send(RequestType.CREATE, block);
	}
	
	public void delete(Block block) throws ConnectException  {
		send(RequestType.DELETE, block);
	}
	
				...
	
	private void send(int ReqType, Block block) throws ConnectException  {
		int triedCount = 0;
		
		while(true) {
			try {
				clientSocket = new Socket(blockInfo.getHostName(), blockInfo.getPort());
				System.out.println("O cliente se conectou ao servidor na porta " 
				+ blockInfo.getPort());
				
				ByteArrayOutputStream bos = new ByteArrayOutputStream();
				ObjectOutputStream    oos = new ObjectOutputStream(bos);
				
				oos.writeInt(ReqType);
				oos.writeObject(block);
				oos.flush();
				
				OutputStream out = clientSocket.getOutputStream();
				
				out.write(bos.toByteArray());
				
				clientSocket.close();
				
				return;
			} catch(ConnectException | UnknownHostException e) {
			
					...
					
			}
		}
	}
\end{lstlisting}

Quando um método não possui relação com qualquer operação que envolva arquivos, o cliente comunica-se  apenas com o serviço de metadados.
\\

O trecho de Código~\ref{code:clent_dfs1} representa o método responsável pela operação de criar um novo diretório. Obviamente tal operação não envolve nenhum arquivo, logo, o cliente não precisa interagir com o serviço de armazenamento. Como demonstrado na Figura~\ref{fig:vis_sis}, a conexão entre o cliente e o serviço de metadados é feita utilizando-se as facilidades da biblioteca \textit{BFT-SMaRt}, através dos métodos \textit{invokeOrdered} e \textit{invokeUnordered}.
\\
\\
\\

O método \textit{invokeOrdered} é chamado para requisitar operações em que a ordem de execução é relevante. Entretanto, caso as requisições possam ser executadas em qualquer ordem, o método \textit{invokeUnordered} é quem deve ser invocado.
\\

\begin{lstlisting}[basicstyle=\ttfamily\footnotesize, frame=single, caption=Exemplo de método da classe ClientDFS, label=code:clent_dfs1]	
	public int createDir(String tgtName) throws ClassNotFoundException, IOException {
		Metadata metadata = new Metadata(System.currentTimeMillis());
		
		ByteArrayOutputStream out = new ByteArrayOutputStream();
		ObjectOutputStream    oos = new ObjectOutputStream(out);
		
		oos.writeInt(RequestType.CREATEDIR);
		oos.writeObject(currPath.toString());
		oos.writeObject(tgtName);
		oos.writeObject(metadata);
		oos.writeLong(System.currentTimeMillis());
		oos.flush();
		
		byte[]  bytes   = this.proxy.invokeOrdered(out.toByteArray());
		
		ByteArrayInputStream in  = new ByteArrayInputStream(bytes);
		ObjectInputStream    ois = new ObjectInputStream(in);
		
		int result = ois.readInt();
		currDir    = (DirEntries)ois.readObject();
		
		if(result == ResultType.FAILURE) {
			currPath = currDir.getPath();
		}
		
		return result;
	}
\end{lstlisting}

Quado um método envolva alguma operação relacionada com arquivos, é necessário que o Cliente comunique-se tanto com o serviço de metadados quanto o de armazenamento. O método representado pelo trecho de Código~\ref{code:clent_dfs2} é responsável operação de criação de um arquivo, por isso ele é chamado de \textit{create}.
\\

Como dito anteriormente, a comunicação entre o cliente e o serviço de metadado é realizada utilizando a biblioteca  \textit{BFT-SMaRt}. Contudo, a comunicação entre o cliente e o serviço de metadados é feita utilizando uma conexão \textit{Java Socket}, implementada pela classe \textit{ClientServerSocket}.
\\
\\
\\
\\
\\
\\
\\

\begin{lstlisting}[basicstyle=\ttfamily\footnotesize, frame=single, caption=Exemplo de método da classe ClientDFS para operação sobre arquivos, label=code:clent_dfs2]	
	public int create(String fileName, String tgtName) throws ClassNotFoundException, IOException  {
		int result = ResultType.FAILURE;
		try {
				...
		
			byte[]  bytes   = this.proxy.invokeOrdered(out.toByteArray());
			
				...
					
			ClientServerSocket[] css = new ClientServerSocket[nServers];
			byte [] buffer = new byte[blockSize];
			
			switch(raidType) {
				case(RaidType.RAID0):
				for(int i=0; i<nServers; i++) {
					Arrays.fill(buffer, (byte) 0);
					
					bInfo = bList.get(i);
					
					bis.read(buffer, 0, blockSize);
					
					Block block = new Block(bInfo.getID(),buffer);
					
					
					css[i] = new ClientServerSocket(bInfo, block, Option.CREATE, verbose);
					css[i].start();   
				}
				for(int i=0; i<nServers; i++) {
					css[i].join();
				}
				for(int i=0; i<nServers; i++) {
					if(css[i].failure()) {
						failure(bList.get(i));
						result = ResultType.FAILURE;
						break;
					}
				}
				break;
				
				case(RaidType.RAID1):
						
			...
		
				case(RaidType.RAID5):
				
			...
			
			}
		
		...
		
	}
			
\end{lstlisting}

\subsection{Executando o Sistema}
O código-fonte de todo o projeto pode ser encontrado no repositório do \textit{GitHub}, no seguinte endereço:  \href{https://github.com/diogoAF/tccRAID}{https://github.com/diogoAF/tccRAID}. Vale ressaltar que todas as bibliotecas necessárias, incluindo o  \textit{BFT-SMaRt}, já estão incluídas no projeto. Para a execução do sistema são necessárias ao menos oito máquinas remotas, sendo três delas para o serviço de metadados, quatro para o serviço de armazenamento (no caso do RAID 0 ou 1, devido a sua natureza, é possível utilizar apenas duas máquinas), e ao menos uma máquina executando o Cliente.
\\

O primeiro passo é criar um arquivo denotado por \textbf{host.config} dentro de um diretório chamado \textbf{config}. Localizado no diretório \textbf{bin} existe um modelo de construção deste arquivo. Ele é necessário pois é utilizado pelo \textit{BFT-SMaRt} para determinar o IP e porta de cada réplica. Ao final do arquivo deve-se inserir uma nova linha contendo o ID do servidor, endereço IP e porta pela qual ele vai escutar. Por exemplo, 7001 10.1.1.9 11100.
\\

O segundo passo é inicializar o serviço de metadados, para tal deve-se chamar a classe \textit{server.meta.ServerConsole}. A qual deve receber quatro parâmetros, os quais serão listado a seguir. 
\\

\begin{itemize}
	\item O identificador único da réplica;
	\item O tipo do RAID que será utilizado;
	\item O número de servidores de armazenamento;
	\item "v" ~caso deseje mostrar na tela as informações da execução.
	\item "n" ~caso deseje não executar em modo de teste, um modo que grava os dados em um único arquivo do disco.
\end{itemize}

Porém, antes de executar o comando de inicialização, recomenda-se criar um \textit{script} que inicialize o \textit{BFT-SMaRt} e as outras bibliotecas, para tal, basta seguir o formato apresentado logo abaixo. Por conveniência, doravante vamos supor que o \textit{script} foi criado e chama-se \textit{scriptBftSmart.sh}.

\begin{lstlisting}
java -cp .:lib/BFT-SMaRt.jar:lib/slf4j-api-1.5.8.jar:lib/slf4j-jdk14-1.5.8.jar:
lib/netty-3.1.1.GA.jar:lib/commons-codec-1.5.jar $1 $2 $3 $4 $5 $6 $7 $8 $9
\end{lstlisting}

Com o \textit{script} criado, basta executar o seguinte comando (cada um em cada máquina). Repare que cada máquina irá receber um identificador único, entretanto, o restante dos parâmetros são inalterados. Repare que é na inicialização do serviço de metadados que o nível de RAID é determinado e ele não pode ser alterado em tempo de execução. Neste exemplo o serviço está sendo iniciado para tratar o RAID 0 com quatro servidores de armazenamento em modo normal (ou seja, não é o modo de teste) e mostrando as informações da execução.
\\
\\

\begin{lstlisting}
sh scriptBftSmart.sh server.meta.ServerConsole 0 0 4 vn
sh scriptBftSmart.sh server.meta.ServerConsole 1 0 4 vn
sh scriptBftSmart.sh server.meta.ServerConsole 2 0 4 vn
\end{lstlisting}

Com os servidores de metadados devidamente inicializados, deve iniciar o serviço de armazenamento. A classe que deve ser invocada é a \textit{server.data.ServerConsole}. A qual deve receber dois parâmetros, os quais serão listado a seguir. 
\\

\begin{itemize}
	\item O identificador único da réplica;
	\item "v" ~caso deseje mostrar na tela as informações da execução.
\end{itemize}

No exemplo a seguir, o serviço será instanciado com quatro servidores mostrando as informações de execução na tela.
\\

\begin{lstlisting}
sh scriptBftSmart.sh server.data.ServerConsole 1001 v
sh scriptBftSmart.sh server.data.ServerConsole 1002 v
sh scriptBftSmart.sh server.data.ServerConsole 1003 v
sh scriptBftSmart.sh server.data.ServerConsole 1004 v
\end{lstlisting}

Nesse ponto os servidores de dados já devem ter estabelecido conexão com o serviço de metadados. Desta forma, resta apenas ativar o cliente. Para tal, é possível ativá-lo de dois modos, o real e o de testes. No modo real será apresentado uma interface de terminal onde o usuário poderá interagir informando qual operação ele deseja que o sistema execute. Enquanto que no modo de teste é passado via linha de comando qual operação deve ser executada (\textit{r} para leitura ou \textit{w} para escrita), o tamanho do arquivo de teste (1,100,1000 ou 10000 todos em \textit{kilobytes}), a quantidade de \textit{threads} clientes que serão instanciadas e a quantidade de operações. Os arquivos que são utilizados no modo de teste também podem ser encontrados na pasta \textbf{bin}.
\\

Para executar um cliente em modo real, é necessário chamar a classe \textit{client.ClientConsole}. A qual recebe apenas o identificador do cliente como parâmetro. No exemplo a seguir, é inicializado um cliente de id 7001.
\\

\begin{lstlisting}
sh scriptBftSmart.sh client.ClientConsole 7001
\end{lstlisting}

No caso do cliente em modo de teste, deve-se chamar a classe \textit{client.ClientTest}. A qual possui parâmetros de entrada idênticos aos da classe \textit{client.ClientConsole}. Desta forma, no exemplo a seguir o cliente teste vai instanciar 10 \textit{threads} clientes, onde cada uma irá executar 500 operações de escrita para um arquivo de 1 KB.
\\

\begin{lstlisting}
sh scriptBftSmart.sh client.ClientTest 7001 w 1 10 500
\end{lstlisting}