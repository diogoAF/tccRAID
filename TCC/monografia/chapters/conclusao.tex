Neste capítulo iremos apresentar as conclusões obtidas com os resultados dos experimentos realizados, além de uma breve explanação sobre quais são os rumos que podem ser tomados a fim de evoluir o sistema, quais melhorias podem ser realizadas de modo a garantir um programa mais robusto e confiável.
\\

\section{Conclusões}
Após a analise realizada sobre os dados coletados em nossos experimentos foi possível levantar algumas conclusões sobre o nosso sistema de arquivos distribuídos baseado nos conceitos de RAID.

\section{Trabalhos Futuros}
Visto os resultados obtidos em nossos experimentos, observamos que o sistema comporta, sem diminuir drasticamente o tempo de resposta, cinquenta usuários simultâneos  solicitando diversas operações, pretendemos estudar modos para aumentar essa quantidade, de modo que o nosso sistema possa ser usado em um ambiente real.
\\

Atualmente todo o gerenciamento realizado pelo serviço de metadados ocorre em tempo de execução, sendo salvo apenas em memória, pretendemos modificar o código para que o mesmo possibilite que as informações gerencias sejam armazenadas em disco, memória não volátil. Pois da forma realizada atualmente, toda vez que o sistema é iniciado as informações gerencias estão em branco, ou seja, todos os blocos de arquivos armazenados no serviço de armazenamento não passam de lixo, espaço de disco desperdiçado. Quando o serviço de metadado estiver integrado com algum tipo de banco de dados esses dados serão preservados mesmo quando os servidores precisarem ser desligados por qualquer motivo
\\

Durante a realização de experimentos, foi observada a ocorrência de algumas exceções durante a execução do programa nas máquinas remotas de servidores de armazenamento. Essas exceções foram de \textit{OutOfMemoryError: Java heap space}, causado pela possível falta de memória disponível no JVM, a máquina virtual de java, e de \textit{OutOfMemoryError: GC Overhead limit exceeded}, a exceção que ocorre quando o coletor de lixo de java consome a maioria de tempo de execução desalocando componentes desnecessários da memória, impedindo que o programa prossegue a sua tarefa.
Durante experimento elas foram evitadas mudando a configuração básica do JVM, aumentando a capacidade máxima da memória que pode utilizar. Contudo, como não podemos considerar que todos os computadores são equipados com memória suficiente para executar o nosso sistema, é necessário que seja feito a redução de uso da memória por sistema. 
Como neste trabalho foi focalizado principalmente em construção do sistema, o código fonte do programa ainda pode ser revisado e otimizado, assim, conseguindo o melhoramento em relação ao uso da memória e, se for possível, o aumento no desempenho do sistema todo.
\\


