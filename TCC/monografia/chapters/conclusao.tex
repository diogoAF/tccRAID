Este capítulo apresenta as conclusões obtidas com os resultados dos experimentos realizados. Além de possuir uma breve explanação sobre quais são os rumos que podem ser tomados a fim de evoluir o sistema, quais melhorias podem ser realizadas de modo a garantir um programa mais robusto e confiável.
\\

\section{Conclusões}
Este trabalho descreve a criação dum sistema de arquivos distribuídos tolerante à falhas, o qual mantém a confiabilidade dos arquivos armazenados usando os conceitos de RAID, e o serviço fornecido pelo sistema é protegido através da biblioteca \textit{BFT-SMaRt}. Os resultados obtidos ao longo dos experimentos demonstram que o RAID 5 tem melhor eficiência para transferência de dados se comparado ao RAID 1, apesar do custo extra do cálculo de geração dos blocos de paridade. Por focalizar no desempenho, sem qualquer preocupação sobre a segurança dos arquivos, o RAID 0 apresenta maior taxa de transferência de dados.
\\

Os resultados comprovam as vantagens de adotar os concentos de RAID na construção dum sistema de arquivos distribuídos,tanto em sistemas focados no desempenho quanto para os que se preocupam com a segurança dos dados armazenados. Nota-se que o RAID 5 é uma opção viável para sistemas que necessitam satisfazer condições de desempenho razoáveis, possibilitando manter a segurança dos dados armazenados.
\\


\section{Trabalhos Futuros}
Neste trabalho foram realizados experimentos com o objetivo de realizar a comparação de desempenho para armazenamento de arquivos entre três níveis de RAID, dando ênfase nas operação de leitura e escrita. Contudo, ainda podem ser realizados outros tipos de experimentos. Verificar o que acontece quando aumentam-se os números de servidores em operação nos serviços de metado e armazenamento. Ou comparar a eficiência de recuperação de um arquivo entre os RAID 0, 1 e 5. Por fim, pode-se estender o sistema para suportar outros níveis de RAID, como por exemplo o RAID 50.
\\

Atualmente todo o gerenciamento realizado pelo serviço de metadados ocorre em tempo de execução, sendo salvo apenas em memória, pode-se modificar o código para que o mesmo possibilite que as informações gerências sejam armazenadas em disco, memória não volátil. Pois da forma realizada atualmente, toda vez que o sistema é iniciado as informações gerências estão em branco, ou seja, todos os blocos de arquivos armazenados no serviço de armazenamento não passam de lixo, espaço de disco desperdiçado. Quando o serviço de metadados estiver integrado com algum tipo de banco de dados essas informações serão preservadas mesmo quando os servidores precisarem ser desligados por qualquer motivo.
\\

Visto os resultados obtidos nos experimentos, observou-se que o sistema comporta, sem diminuir drasticamente o tempo de resposta, cinquenta usuários simultâneos solicitando diversas operações. Pode-se estudar modos afim de aumentar essa quantidade, de modo que o sistema possa ser usado em um ambiente real. Além disso, na realização dos experimentos foram observadas ocorrências de algumas exceções Java durante a execução do programa nas máquinas remotas de servidores de armazenamento. As exceções foram de \textit{OutOfMemoryError: Java heap space}, causada pela falta de memória disponível para a máquina virtual java (ou JVM). Outra exceção foi a \textit{OutOfMemoryError: GC Overhead limit exceeded}, exceção que ocorre quando o coletor de lixo de java consome a maioria do poder de processamento desalocando componentes desnecessários da memória, deste modo impedindo que o programa prossiga com a sua tarefa principal.
\\

Durante os experimentos,tais exceções foram evitadas modificando as configurações básicas da JVM. Modificações como o aumento da capacidade máxima da memória alocada pela JVM. Contudo, como não é possível assegurar que qualquer computador seja capaz de aumentar a quantidade de memória alocada para a JVM, é necessário otimizar o gerenciamento da memória pelo programa.
\\


