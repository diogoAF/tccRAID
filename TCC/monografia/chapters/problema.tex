
	\section{Problema}
	O \textit{overhead} de espaço indica a taxa de espaços ocupados na unidade de armazenamento pela parte redundante dos arquivos, que são considerados espaços inativos por não armazenar os dados efetivos.
	É obtido com razão entre a quantidade de partes redundantes gerados e a quantidade de partes originais do arquivo 
	podendo mostrar também a estimativa para quanto vai crescer de tamanho quando foram criadas as partes redundantes para os arquivos armazenados no momento.
	Na maioria dos sistemas de arquivos distribuídos que encontra hoje em dia, para assegurar a alta confiabilidade nos arquivos armazenados, incorporam o método de replicação distribuindo três ou mais cópias de segurança ao longo dos servidores que constituem o sistema. Porém esse método apresenta \textit{overhead} de espaço de 200\%, no mínimo, necessitando de triplo de espaço que os arquivos realmente iriam ocupar.\\
	
	A consequência esperada por aumento de capacidade de armazenamento total necessária do sistema é indução para incremento de escala do mesmo, o que resulta em aumento de custo.  
	

	
	
	

	
	
	
	

 

	

