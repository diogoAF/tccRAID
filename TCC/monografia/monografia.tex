\documentclass[bacharelado]{unb-cic}
\usepackage[american,brazil]{babel}
\usepackage[T1]{fontenc}
\usepackage{indentfirst}
\usepackage{natbib}
\usepackage{xcolor,graphicx,url}
\usepackage[utf8]{inputenc}
\usepackage{listings}
\usepackage{color}

\definecolor{dkgreen}{rgb}{0,0.6,0}
\definecolor{gray}{rgb}{0.5,0.5,0.5}
\definecolor{mauve}{rgb}{0.58,0,0.82}

\lstset{frame=tb,
	language=sh,
	aboveskip=3mm,
	belowskip=3mm,
	showstringspaces=false,
	columns=flexible,
	basicstyle={\small\ttfamily},
	numbers=none,
	numberstyle=\tiny\color{gray},
	keywordstyle=\color{blue},
	commentstyle=\color{dkgreen},
	stringstyle=\color{mauve},
	breaklines=true,
	breakatwhitespace=true,
	tabsize=3
}

%\bibpunct[; ]{(}{)}{,}{a}{}{;}%muda colchetes para parênteses

% definições prévias do documento
\title{Aplicação de RAID em Sistema de Arquivos Distribuídos}

\orientador{\prof \dr Eduardo A. P. Alchieri}{CIC/UnB}

\coordenador{\prof \dr Rodrigo Bonifacio de Almeida}{CIC/UnB}

\diamesano{18}{Junho}{2016}

                           
\membrobanca{\prof\dr Professor I}{CIC/UnB}

\membrobanca{\prof \dr Professor II}{CIC/UnB}

\autor{Diogo Assis}{Ferreira} 
\coautor {Getúlio Yassuyuki}{Matayoshi}


\CDU{004.4}

\palavraschave{RAID, BFT SMaRt, Sistema de arquivo distribuído}
\keywords{RAID, BFT SMaRt, Distribuited file sistem}


\begin{document}

\maketitle

\pretextual

\begin{agradecimentos}
Foi uma tarefa árdua chegar ao momento atual, monografia completa, créditos conquistados, colação marcada. Digo que a missão foi complicada pois exigiu dezenas de horas de estudo, noites em claro, programas que não compilavam, \textit{segmentation fault} aparentemente sem explicação, cálculos que não batiam, diversos finais de semana na faculdade. Porém nenhuma dessas frustrações supera a satisfação de conseguir superar tais problemas, a conta realizada com precisão, o programa otimizado que executa com exatidão, a prova com sabor de vitória, tudo levando àquela bela e merecida noite de sono antes do ciclo recomeçar. Por isso digo que não foi fácil, mas foi enriquecedor, o quanto aprendi, o quanto vivenciei, são experiências preciosas que carregarei com carinho e orgulho. Entretanto, devo ressaltar que essa trajetória não foi percorrida por mim sozinho, em todo o momento eu recebia direta ou indiretamente o apoio de pessoas importantes. Meu pai, sempre disposto a preparar uma bela refeição entre meus estudos, minha mãe, preocupada todas as vezes em que ficava até tarde na UnB, minha irmã, sempre pronta pra me pentelhar (e ajudar a aliviar todo o estresse dos estudos), Talita, minha namorada, sempre disposta a me receber com carinho e confortar todas minhas (várias) lamurias, minha família: avôs, tias, primos também dando seu suporte da forma que podiam, amigos e colegas que passavam os momentos de desespero e diversão na faculdade ou fora dela, sou muito grato a todos. Contudo, gostaria de expressar minha gratidão em especifico à meu orientador que teve muita paciência, boa vontade e bom humor para me ajudar nessa tortuosa parte final do curso e a meu parceiro, Getúlio, que foi de fundamental importância para a conclusão deste trabalho, obrigado! Obrigado a todos! E até o próximo desafio!
\\

Diogo Assis Ferreira
\\

\end{agradecimentos}

\begin{resumo}
Este trabalho tem como foco a construção de um sistema de arquivos distribuídos que é tolerante a falhas, sem prejuízo de desempenho e que consuma menor quantidade de \textit{overhead} do que o sistema apresentado pelo método de replicação de dados tradicional. A fim de alcançar tal objetivo foram utilizadas as características inerentes ao conceito do RAID 0, RAID 1 e RAID 5, as quais possibilitaram o teste das latências e o teste do \textit{throughput} das operações de leitura e escrita com arquivos de quatro tamanhos distintos, sendo esses 1KB, 100KB, 1MB e 10MB. A coleta e a análise dos dados resultou na conclusão de que o sistema alcança o objetivo proposto, sendo que, quando executado sob as parametrizações do RAID 5, o desempenho geral é superior se comparado aos outros dois modelos de RAID implementados.
\\

\end{resumo}

\selectlanguage{american}

\begin{abstract}
This work has as its focus the construction of a fault tolerant distributed file system, without compromising performance, that consumes a lesser quantity of overhead than the system presented by the traditional data replication method. In order to achieve this goal, inherent characteristics of the concept of the RAID 0, RAID 1 and RAID 5 were used, which enabled the test of the latencies and the test of the throughput of the reading and writing operations with four distinct file sizes, these being 1KB, 100KB, 1MB and 10MB. The gathering and the analysis of these data resulted in the conclusion that the system reaches the proposed objective, being that, when run under the parameterizations of the RAID 5, the overall perfomance is superior if compared to the other two implemented RAID models.
\\

\end{abstract}

\selectlanguage{brazil}
\tableofcontents
\listoffigures
\listoftables

\textual

\chapter{Introdução}

Este capítulo sintetiza uma pequena motivação do porquê um sistema de arquivos distribuído tolerantes a falhas é de fundamental importância para que um sistema distribuído possa ser considerado robusto e confiável, além de apresentar motivos para a relevância de um sistema distribuído como um todo. Utilizando a contextualização como base, ainda é abordado o objetivo geral e os específicos que desejamos alcançar neste trabalho. Por fim, um breve sumário sobre a disposição dos próximos capítulos deste trabalho é apresentado.
\\

 

	



	\section{Contextualização}
	Tradicionalmente a ciência era dividida em três paradigmas: empírica, lógica e computacional, porém o cientista computacional Jim Gray idealizou um quarto chamado de \textit{data-intensive science}~\cite{hey2009}. Tal paradigma caracteriza-se pelo intenso processamento de \textit{big data} afim de conseguir informações úteis. \textit{Big data} consiste de um conjunto extremamente amplo de dados não estruturados. Justamente pela quantidade massiva de dados que compõe o \textit{big data}, é impraticável utilizar computadores convencionais para seu processamento.
	\\
	
	Atualmente a taxa de dados processados e armazenados globalmente aumenta anualmente. Tal crescimento não foi influenciado apenas pelos estudos científicos e grandes empresas, visto que grande parcela da população criou o abito de acessar diariamente as mais diversas modalidades de redes sociais, além de utilizar as facilidades da computação nas nuvens disponibilizada por empresas como o Dropbox ou a Google. Como exemplo da popularização de tais serviços, o \textit{Dropbox} divulgou possuir mais de 300.000.000 usuários cadastrados~\cite{dropbox} em 2014, onde cada usuários dispõem um espaço mínimo de 2GB para armazenar seus arquivos remotamente, desta forma, para suprir as necessidades de seus clientes, são necessários no mínimo 600PB de espaço de armazenamento. Sobre o ramo das redes sociais, existe o caso do \textit{Facebook}, considerada em 2012 como a mario em seu ramo de atuação, cujos usuários produzem cerca de 600TB diariamente~\cite{facebook14}. Enquanto que na área de pesquisas, A Organização Europeia para a Pesquisa Nuclear (\textit{The European Organization for Nuclear Research - CERN}), o maior laboratório de física de partículas do mundo, informou em 2013 que ao longo de 20 anos produziu 100PB de dados em seus experimentos~\cite{cern}.
	\\
	
	Os casos apresentados anteriormente demonstram o imenso volume de dados que são gerados, processados e armazenados diariamente em todo o globo. Nesses casos em que o volume de dados armazenados chega na ordem de \textit{petabytes} faz-se necessário buscar formas eficientes e seguras para salvar todas as informações. Por exemplo, utilizando-se discos rígidos de 2 TB, serão necessários 50.000 dispositivos para conseguir a capacidade total de 100 PB. Tamanha quantidade de equipamentos seria impossível de serem instaladas em um único servidor, o que acarreta na necessidade de se utilizar um sistema distribuído com vários servidores, ditos como nós, trabalhando conectados através de uma rede. Desta forma, é indispensável a utilização de um sistema de arquivos distribuído (SAD) para o gerenciamento de todos os arquivos espalhados entre todos os servidores. Como exemplos reais de SAD, temos  o \textit{Google File System}, desenvolvido e utilizado pela própria \textit{Google}, O \textit{Apache Hadoop}, utilizado pelo \textit{Facebook},além do \textit{Amazon S3}, utilizado pelo \textit{Dropbox}.
	\\
	
	Visto que em um SAD os arquivos estão armazenados distribuidamente entre vários nós, a falha de um nó pode inviabilizar o uso do sistema como um todo. Tal situação ocorre pelo fato de que os arquivos armazenados pelo nó faltoso ficam inacessíveis para o restante dos nós do sistema, deste modo inviabilizando a recuperação de alguns ou até mesmo todos os dados contidos no SAD. Somado a isso, ainda temos o agravante de que a probabilidade de ocorrerem falhas em algum dos equipamentos cresce proporcionalmente com o aumento dos componentes envolvidos, desta forma, o incremento de escala pode resultar na redução da confiabilidade no sistema.
	\\
	
	É fundamental encontrar formas de contornar os problemas enfrentados pelos sistemas distribuídos, de modo a torná-lo tolerante à falhas. Para o caso de dados inacessíveis, o método de maior simplicidade é o de replicação. Na replicação os dados contidos em um nó são copiados integralmente para outros nós, assim criando cópias de segurança chamadas réplicas. Deste modo, mesmo quando algum nó apresentar problemas, o sistema ainda será capaz de recuperar qualquer arquivo que estivesse armazenado no nó defeituoso. O grande gargalo dessa abordagem é o desperdício de espaço, visto que cada réplica irá consumir a mesma quantidade de espaço de armazenamento que o arquivo original. Em sistemas que lidam com \textit{big data}, a redundância total de dados muitas vezes é impraticável.
	\\
		
	\section{Justificativa}
	No contexto dos sistemas distribuídos o conceito de paralelismo é muito presente, característica essa que acarreta no aumento de máquinas conectadas pela rede e consequentemente a probabilidade de alguma dessas máquinas sofrer erros de \textit{hardware} ou \textit{software}. Quando qualquer falha dessa natureza ocorrer, é necessário que contra medidas tenham sido implementadas a fim de minimizar os danos. Uma dessas medidas é que o sistema de arquivos distribuído seja tolerante a falhas, tolerância essa que pode ser alcançada utilizando-se a replicação total dos dados, entretanto esse \textit{overhead} ocupa espaços de armazenamento que poderiam ser utilizados para novos dados.
	\\
	
	Os conceitos de RAID podem ser utilizados a fim de aumentar o desempenho e a confiabilidade de um sistema. Durante o planejamento da tecnologia RAID vários pontos foram levados em consideração, um deles foi como aumentar a taxa de transferência da unidade de armazenamento. O modo escolhido foi utilizar o paralelismos. Com um conjunto de discos onde cada um deles executa de forma independente e paralela aos outros, é possível realizar acesso simultâneo em vários discos do conjunto, assim aumentando a vazão total de dados transferidos e resultando no aumento de desempenho do sistema. Enquanto que na área de segurança, exitem distintas formas de proteção dos dados baseadas na redundância, tais como espelhamento, replicação e paridade.
	\\
	
	Os pontos fortes da tecnologia RAID podem ser extrapoladas para um sistema de arquivos distribuídos a fim de suprir suas fragilidades sobre proteção de dados e problemas de confiabilidade.
	\\


\section{Objetivo}

O nosso objeivo neste artigo é a construção de um sistema de arquivos distribuído baseado na computação em nuvem que possui menor \textit{overhead} de espaço, comparado a 200\% de método de replicação, sem degradar muito o performance e a confiabilidade na armazenamento de dados. \\

Para isso, nós observamos em técnica de redundância usado na tecnologia RAID, onde tem possibilidade de diminuir o espaço ocupado pela parte redundante com atribuição de paridade nos arquivos, ao invés de simples replicação. Dessa forma, é possível de reduzir o \textit{overhead} de espaço até 50\%, quando for gerada uma paridade para cada duas partes de arquivo original, ou até menos, dependendo da configuração utilizada para calcular as paridades.
Além de otimização de uso de espaço para armazenamento, também é esperado a melhoria na distribuição de caga entre os servidores. Como os arquivos estão fragmentados em blocos e espalhados em todos os servidores, a probabilidade de ocorrer a concentração de acesso em um servidor tende a ser reduzida.
Também notamos no fato de que a tecnologia RAID foi inicialmente desenvolvido para aumentar a taxa de transferência da unidade de armazenamento, mantendo a mesma velocidade de leitura/estrita. Isso foi realizado com leitura/escrita simultânea em vários discos rígidos, aumentando o tamanho total de dados transferidos por vez numa operação de entrada/saída. Esta ideia também foi incorporado no nosso sistema, uma vez que no sistema distribuído é requisito básico o uso de vários servidores para trabalhar em conjunto.\\

Para gerenciar o armazenamento de forma eficientemente, todos os arquivos são divididos em duas partes, o metadado, as informações sobre arquivo, e o conteúdo, o dado em si do arquivo. 
Assim, no conjunto de servidor vai ter dois tipos diferentes de nós, alguns que armazena os metadados dos arquivos e outros que armazena o conteúdo dos arquivos.
A confiabilidade do conteúdo do arquivo é assegurado com conceito de RAID, e da mesma forma os metadados serão protegidos através de BFT-SMaRt, uma biblioteca JAVA que fornece a tolerância a falha, realizada com uso de método de replicação do serviço.\\

%Assim, a nossa abordagem está focalizado em construção de uma forma eficiente para armazenar os arquivos com segurança no sistema de arquivos distribuído.

Também será feito a comparação de desempenho entre os dois sistemas, um que usa método de replicação e outro que usa método de RAID.
Comparando os dois métodos de redundância, será possível a visualização das vantagens e desvantagens que existem entre eles.












	\section{Organização de capítulos}
	
	Os capítulos estão organizados de seguinte forma. 
	No capítulo 2 apresenta os conceitos sobre sistema de arquivos distribuído, explicando as principais características que esse tipo de sistema apresenta. 
	No capítulo 3 descreve a computação nas nuvens. 
	No capítulo 4 apresenta os princípios de BTF-SMaRt, uma biblioteca que realiza a replicação de serviços para prover a tolerância a no sistema, assim como os protocolos centrais e sua implementação.
	No capitulo 5 apresenta a funcionalidade de tecnologia RAID.
	No capitulo 6 descreve como vai ser desenvolvido o sistema proposta.
	%fundamentais utilizados para a cnstrução do sistema, como , computação na nuvem, BFT-SMaRt e RAID. 
\chapter{Sistema de arquivos distribuído}


 

	



	%\section{Sistema de arquivos distribuído}
	
	\section{Sistema distribuído}
	Hoje em dia, podem ser encontrada diversas tarefas computacionais que são impossíveis de ser processadas por um computador comum, devido a simples limitação da capacidade de processamento que uma máquina pode atingir. A solução tipicamente tomada por muitas organizações é o uso de sistema distribuído, no qual possui algumas vantagens comparado a um sistema centralizado, como pode ser visto a seguir.
	
	\begin{itemize}
		\item Melhor relação custo-benefício;
		\item Capacidade de processamento além dos limites práticos de sistema centralizado;
		\item Alta confiabilidade e disponibilidade;
		\item Relação linear entre crescimento de custo e desempenho.
	\end{itemize}
	Os sistemas de computação distribuída são caracterizados pela sua estrutura, cuja consiste em interação entre um grande número de dispositivos que executam seus próprios programas, mas que são afetados recebendo as mensagens ou observando a memória compartilhada de outros dispositivos.
	\\
	
	Existem várias definições e pontos de vista sobre o que são sistemas distribuídos. Coulouris~\cite{coulouris06} define um sistema distribuído como "um sistema no qual os componentes de hardware e software localizadas nos computadores conectados por rede comunicam e coordenam suas ações somente por passagem de mensagens", e Tanenbaum~\cite{tanenbaum07} define como "uma coleção de computadores independentes que aparenta ao usuário ser um computador único". Leslie Lamport, um famoso pesquisador sobre sistema distribuído, disse uma vez que "um sistema distribuído é aquele em que é impedido de prosseguir com seu trabalho devido a uma falha de algum computador que nunca ouviu falar" refletindo o grande número de desafios enfrentados pelos projetistas de sistemas distribuídos.
	\\
	
	Assim, quando um sistema distribuído aumenta o seu tamanho, torna-se cada vez mais difícil de prever ou mesmo entender o seu comportamento. A parte da razão para isso é simples falta de ferramentas que gerenciam a complexidade e outra parte é que um sistema distribuído de grande porte traz consigo a quantidade enorme de não-determinismo inerente, os eventos imprevisíveis, como atrasos nas chegadas de mensagem, súbito falhas de componentes, ou em caso extremo, as ações nefastas de defeitos ou máquinas maliciosos que agem opostamente para os objetivos do sistema como um todo.
	
	\section{Sistema de arquivos distribuído}
	
	O sistema de arquivos distribuído (SAD) é um tipo de sistema distribuído que tem proposta de permitir que usuários de computadores fisicamente distribuídos compartilhem dados e recursos de armazenamento usando um sistema de arquivo comum~\cite{levy90}.
	
	\subsection{Transparência}
	A transparência é uma das principais propriedades que os sistemas distribuídos em geral apresentam. Um sistema distribuído transparente é aquele que se mostra como se fosse apenas um único sistema para usuários e aplicações. O conceito de transparência pode ser aplicado diversos aspectos de um sistema distribuído, os mais importantes são mostrado a seguir~\cite{tanenbaum07}.	

	\subsubsection{Transparência de acesso} não necessita fornecer a localização dos recursos, ou seja, os programas devem executar os processos de leitura/escrita de arquivos remotos da mesma maneira que operam sobre os arquivos locais, sem qualquer modificação no programa. O usuário não deve perceber se o recurso acessado é
	local ou remoto.
	
	\subsubsection{Transparência de localização} os programas clientes devem ver um espaço de nomes de arquivos uniforme, sem a necessidade de fornecer a localização física dos arquivos para encontrá-los, mesmo que esses arquivos se desloquem entre os servidores.
	
	\subsubsection{Transparência de migração} independente dos arquivos se moverem entre servidores, os programas clientes não precisam ser alterados para a nova localidade do grupo de arquivos. Essa característica permite flexibilidade em mover arquivos sem comprometer toda a estrutura, ou ter que refazer links entre programas clientes e o local do arquivo.
	
	\subsubsection{Transparência de desempenho} o desempenho da aplicação cliente não poderá ser comprometido enquanto ocorre uma variação dos processos sobre os recursos disponíveis pelos SADs, isto é, mesmo que haja concorrência no acesso pelos arquivos isso não deve afetar os usuários.
	
	\subsubsection{Transparência de escalabilidade} os recursos computacionais podem sofrer alterações para abrigar maior poder computacional ou o ingresso de novos servidores sem prejudicar o serviço.
	
	\subsubsection{Transparência a falhas} garantir a disponibilidade dos arquivos ininterruptamente e se ocorrerem falhas o programa cliente não deverá saber como elas serão tratadas.
	
	\subsubsection{Transparência de replicação} várias cópias dos mesmos arquivos armazenados em locais diferentes para garantir a disponibilidade. A aplicação cliente deverá
	visualizar apenas uma cópia do mesm não necessitando saber a quantidade replicada e o local.
	\\
	
	A transparência é altamente desejável em sistemas distribuídos, mas nem sempre é possível alcançá-la ou, em determinadas situações, não convém ocultá-la. Pode-se destacar uma situação que seja mais conveniente o usuário tomar uma decisão sobre
	alguma falha do que o sistema distribuído tentar resolver por si só. Isso pode ser observado quando um serviço, por repetidas vezes tenta, estabelecer uma comunicação com o servidor na internet, neste caso o melhor é informar ao usuário sobre a falha e que ele tente mais tarde
	
	\subsection{Características}
	
	Vários SADs são utilizados para resolver diferentes tipos de problemas, portanto as características de um sistema variam dependendo de requisitos do sistema. Alguns sistemas dão mais importância na taxa de transferência e outros em manter a consistência dos arquivos, por exemplo. Entretanto, em geral a maioria dos sistemas devem ter as características essenciais para lidar com as seguintes questões a saber~\cite{tanenbaum07,coulouris06, galli00, kon94}.
	
	%\begin{itemize}
	%\item Disponibilidade
	\subsubsection{Disponibilidade}
	A maioria dos SADs possuem serviços dependente da rede interna ou externa, onde ainda continua sendo um meio relativamente instável, podendo deixar o serviço fora de ar por causa de algum inconveniente que ocorreu na rede. A sua arquitetura que possui grande quantidade de computadores constituindo o sistema também é um dos fatores que causam a indisponibilidade do serviço, pois a probabilidade de acontecer uma falha cresce de acordo com aumento do número de entidades envolvidos. Foram feito vários estudos para evitar que os serviços deixem de ser oferecidos, independente de qual for a razão da inconveniência. Assim, um SAD deve implementar um esquema para manter o acesso aos seus recursos de qualquer jeito, deixando sempre disponíveis para responder às requisições enviados por usuários. A falha que acontece em alguns servidores não pode influenciar a execução de serviço do sistema. Além disso, o usuário não precisa saber como tal esquema funciona e nem como foi implementado, podendo continuar usar o sistema sem preocupar com possíveis transtornos que estão acontecendo. Essa restrição é incorporado para manter a propriedade de transparência do sistema.
	
	%\item Operações atômicas
	\subsubsection{Operações atômicas}
	A operação atômica é um conjunto de operações que aparece para resto do sistema como se fosse ocorrido instantaneamente.
	Quando um arquivo é submetido a este tipo de operação, o estado dele muda de um para outro sem apresentar outros estados intermediários durante a execução.
	Consequentemente, uma operação sobre arquivo é considerado atômica somente se as etapas que ela segue não são reconhecidas por outros processos que estão fora desta operação~\cite{tanenbaum07_2}. 
	Assim, para uma operação ser atômica deve satisfazer as seguintes condições.
	
	\begin{enumerate}
	\item Enquanto as etapas da operação estejam em progressão, nenhum entidade externo consegue perceber o estado intermediário desta operação.
	\item Uma operação é efetuada somente se todas as etapas dela são concluídas com sucesso, caso contrário, se tiver uma etapa que falhou, todas as etapas serão abortadas e o estado do sistema volta para antes de executá-la.
	\end{enumerate}
	
	Geralmente as operações de leitura, escrita, criação ou remoção de um arquivo apresentam propriedade atômica para maioria dos SADs.\\
	
	As transações são mecanismos que permitem realizar uma sequência de operações de forma atômica. Tais mecanismos disponibilizam determinados comandos para os usuários para que eles possam escolher quais operações serão executadas dentro de transações. Para montar uma transação, existem os comandos início e fim. O comando de início avisa ao sistema que todas as operações a partir daquele ponto estarão dentro da transação, e o comando de finalização indica que não virá mais nenhuma operação para aquela transação.
	Assim, caso alguma dessas operações falhem, o sistema desfaz, ou aborta, todas as alterações que as operações antes daquela realizaram. Isso é chamado de \textit{rollback} ou \textit{abort}. Caso todas as operações sejam executadas sem problemas ou erros, ao chegar no fim da transação é realizado um \textit{commit}, ou seja, todas as alterações que foram executadas são efetivadas e persistidas, de tal forma que outros processo possam percebê-las. Com isso as transações implementam a semântica do tudo ou nada, ou seja, ou todas as operações são executadas com sucesso, ou nenhuma será executada. Isso faz das transações um importante mecanismo de tolerância a falhas, pois elas evitam que pequenas falhas prejudiquem a integridade de todo o sistema.
	
	%\item Replicação de arquivos
	\subsubsection{Replicação de arquivos}
	No contexto de SAD, a abordagem de replicação tem dois motivos para ser utilizado. 
	Um deles é a distribuição da carga de acesso entre todos os servidores que constitui o sistema. 
	Se existir um arquivo que é acessado por clientes com alta frequência, a transmissão concentra no servidor que armazena tal arquivo, causando sobrecarga neste servidor e consumo excessivo da banda dessa conexão, o que resulta na degradação de desempenho do sistema todo. 
	Para evitar a concentração de acesso, a solução trivial que foi abordada é criar algumas cópias de arquivos, geralmente três, e distribuir ao longo dos servidores para descentralizar os acessos. \\
	
	
	Além disso, se um sistema de arquivos oferece essa funcionalidade, a confiança do serviço de arquivos é generosamente aumentada.
	Caso um determinado servidor caia ou fique indisponível, o serviço de arquivos ainda pode continuar com suas operações por possuir cópias dos dados em outro ponto da rede.
	Assim, replicação de arquivos provê tolerância a falhas, já que o usuário nem sequer percebe que servidor que ele estava usando caiu e que outro entrou no lugar para prover o recurso que ele estava usando. Por isso o sistema também deve oferecer transparência de replicação pois o usuário não precisa saber como o sistema cuida da replicação desse arquivo.
	O maior problema nessa característica do SAD é que a implementação pode ser muito complicada, pois é necessário manter os dados sincronizados e coerentes ao mesmo tempo.
	Existem dois tipos de implementações: a primeira utiliza comunicação em grupo, que consiste em quando ocorrer uma alteração por algum dos servidores, este manda por \textit{broadcast} para os outros servidores dizendo que o arquivo foi alterado. Estes, por
	sua vez, podem alterar esse arquivo imediatamente ou somente quando forem utilizá-lo; a segunda utiliza votação e números de versão. Isso significa que quando um cliente pedir permissão para alterar um arquivo, os servidores votarão entre eles pra saber quem possui a versão mais recente, e esse servidor será o servidor padrão daquele arquivo, e seu número de versão será incrementado. Todas essas ideias, além de serem complicadas de implementar, geram alguns problemas. Manter a sincronização entre os servidores, para o caso de alterações no sistema de arquivos, é uma delas
	
	%\item Tolerância a falhas e gerenciamento de falhas
	\subsubsection{Tolerância a falhas e gerenciamento de falhas}
	O sistema deve ser capaz de continuar com sua operação normalmente mesmo que um ou mais componentes apresentam a falha, isolando esses componentes.
	Durante a transmissão dos dados entre servidores e clientes, podem ocorrer falhas, seja por excesso de tráfego de pacotes pela rede, seja por algum dos servidores estar sobrecarregado. Além disso, podem ocorrer falhas de hardware, especialmente dos mecanismos de armazenamento, de transmissão, etc. Esses problemas acontecem em grande parte porque os sistemas distribuídos são implementados sobre redes de computadores que não são totalmente confiáveis. É por causa disso que a maior parte da complexidade de sua implementação está em levar em conta esse ambiente propício a falhas. Um sistema distribuído precisa usar um protocolo de comunicação com capacidade para detecção de erros de transmissão~\cite{kon94}. Assim, caso uma mensagem chegue alterada no seu destino, o protocolo precisa perceber isso e retransmiti-la. Isso deve ocorrer também para mensagens que se perderam no caminho.
	Um outro problema que a rede pode ter é o seu particionamento por tempo indeterminado. Mas não é só com a rede que devemos nos preocupar. O \textit{hardware} dentro das máquinas também pode apresentar falhas. Por exemplo, um disco rígido pode deixar de funcionar de um momento para outro, seja por excesso de uso ou até mesmo por descargas elétricas. Nesse caso pode-se criar soluções desde redundância física do equipamento, realizada via hardware, ou redundância controlada pelo próprio sistema distribuído, que cuidaria de replicar os dados, para evitar a perda dos mesmos.\\
	 
	Seja qual for o problema, o sistema deve evitar que o cliente fique aguardando uma resposta por muito tempo, ou que seus dados sejam danificados ou até mesmo perdidos. Isso significa que o serviço precisa ter disponibilidade e confiabilidade.
	Porém, muitas vezes essas características se conflitam. Por exemplo, uma forma de garantir a confiabilidade é implementar redundância dos dados, porém a complexidade que isso gera pode aumentar demais a carga do servidor, comprometendo a disponibilidade, pois as respostas aos clientes seriam mais lentas.
	Outro mecanismo que auxilia a confiabilidade é a transação. Ela evita que o conteúdo de algum arquivo fique em um estado inconsistente caso haja uma queda do servidor ou cliente durante a execução de alguma operação sobre o mesmo.
	\\
	
	Das falhas pode-se destacar os seguintes problemas:
	\begin{itemize}
		\item \textbf{falha por queda}, problema físico ou lógico no servidor, causando o travamento do sistema operacional;
		
		\item \textbf{falha por omissão}, significa o não recebimento das mensagens, quer seja por causa de mensagens não aceitas ou por não enviar uma resposta depois da ação concluída;
		
		\item \textbf{falha de temporização}, ocorre quando uma reposta está fora do intervalo de tempo adequado;
		
		\item \textbf{falha de resposta} consiste da resposta emitida estar incorreta, ou seja, o retorno não condiz com a solicitação;
		
		\item \textbf{falha arbitrária}, também conhecida como falha bizantina, ocorre quando um
		servidor envia mensagens inadequadas, mas que não são consideradas como incorretas.
		Também pode ser associada a um servidor que está atuando maliciosamente
		e, portanto, emitindo respostas erradas de forma proposital.
		
	\end{itemize}
	
	
	%\item Escalabilidade
	\subsubsection{Escalabilidade}
	%- capacidade de aumentar o desempenho do sistema com adição de recurso.
	Os sistemas distribuídos são, em geral, projetados e configurados pensando-se na configuração da rede naquele momento. Pode acontecer dessa rede aumentar, ou seja, dezenas ou centenas de novos nós serem adquiridos e conectados nesse sistema. A menos que se tenha pensado nessa situação no momento do projeto dessa rede, dificilmente um SAD apresentará bom desempenho para servir todos esses clientes após um crescimento tão grande da rede. Vários problemas podem ocorrer, como, por exemplo, quando o servidor responde a um pedido de um cliente. Mantém-se esses dados enviados em cache, para permitir uma rápida resposta caso esse mesmo dado seja requisitado novamente. No caso de se ter muitos clientes, ocorrerá de se ter muitos pedidos diferentes, fazendo com que as tabelas do cache sejam atualizadas com frequência, sem a reutilização dos dados lá contidos. Isso acabará causando uma sobrecarga do sistema, pois por ter muitos clientes, muitas requisições serão feitas em um curto intervalo de tempo, e os dados que estavam no cache rapidamente serão trocados por outros requisitados por outros clientes. Caso se tenha cache do lado dos clientes, ao se alterar um arquivo que está sendo usado por muitas outras máquinas, o servidor terá que avisá-las que o cache local das mesmas está inválido, e todas deverão atualizar com a versão do servidor, causando mais sobrecarga. \\
	
	Por outro lado, caso se tenha estimado que a rede seria muito grande e se tenha distribuído o sistema de arquivos em muitos servidores, fica difícil descobrir onde um arquivo está armazenado fisicamente. Por exemplo, se para abrir um arquivo um cliente tiver que perguntar para cada servidor se ele é o responsável por aquele arquivo, certamente haverá um congestionamento na rede. Caso se tente resolver isso colocando um servidor central para resolver todos os caminhos para os arquivos, indicando a localização do mesmo, tal servidor sofrerá sobrecarga. Um sistema escalável é um sistema que leva em conta esses problemas e tenta evitar sua ocorrência quando o número de clientes aumenta extremamente.
	
	%\item Acesso concorrente
	\subsubsection{Acesso concorrente}
	%- deve ser possível o acesso compartilhado aos recursos.
	Vários usuários podem acessar vários arquivos, ou os mesmos arquivos, sem sofrer danos, perda de desempenho ou quaisquer outras restrições. Isso tudo deve ocorrer sem que o usuário precise saber como o acesso é realizado pelos servidores. Assim, é necessário haver transparência de concorrência.
	O maior problema encontrado nas implementações desse tipo de solução é quanto à sincronização dos arquivos, o que inclui leitura e escrita concorrente. A leitura concorrente pode ser implementada facilmente se não houver escrita concorrente, pois quando um arquivo estiver sendo lido, certamente ninguém poderá escrever nele. Caso também se queira escrita concorrente, deve-se levar em conta que quando um cliente escreve em um arquivo, todos os leitores devem ser avisados que o arquivo foi alterado, e todos escritores precisam tomar cuidado para não escrever em cima das alterações que foram feitas por outros. Assim, ou vale a última alteração, ou os escritores discutem entre si para tentar fazer um “\textit{merge}” das alterações. \\
	
	Para se ter uma idéia de como esse problema é complexo, imagine duas operações bancárias ocorrendo simultaneamente para a mesma conta. Uma delas é um saque de R\$200,00 e outra é um depósito de R\$1000,00. Antes dessas operações, suponha que essa conta possua R\$100,00 de saldo, e também suponha que esse valor esteja armazenado em um arquivo de um SAD desse sistema bancário. Quando o cliente da conta for realizar o saque, a aplicação irá armazenar em memória o valor atual do saldo, assim como acontecerá com a aplicação do outro caixa que estará recebendo o depósito. Esta aplicação, então, irá adicionar ao saldo o valor do depósito, e gravará no arquivo o novo saldo, que será de R\$1200,00. Porém, a primeira aplicação irá subtrair do valor armazenado em memória, que para seu contexto é de R\$100,00, o valor do saque, e gravará o resultado, o valor R\$100,00, no mesmo arquivo, sobrescrevendo o valor lá existente. Dessa forma, o cliente perderia seu depósito.\\
	
	Para evitar esse tipo de problema, as aplicações que operam dessa forma podem agrupar um conjunto de operações no sistema de arquivos como sendo uma única transação, deixando a cargo do sistema operacional gerenciar a melhor forma de executar isso.
	Existem alguns mecanismos para o controle dessa concorrência. Dentre eles, destaca-se o mecanismo de \textit{locks}, por ser o mais amplamente utilizado. Tal sistema de controle de concorrência baseia-se no bloqueio do arquivo que se quer acessar antes de acessá-lo, através de uma chamada ao sistema operacional. Caso um segundo processo queira usar o mesmo arquivo, ele tentará realizar o bloqueio, usando o mesmo comando que o primeiro processo, e o sistema operacional o avisará que esse arquivo já está bloqueado. Cabe ao processo decidir se espera na fila pelo desbloqueio ou se continua seu processamento sem o acesso ao arquivo. Esse desbloqueio é realizado pelo processo detentor do arquivo, através de um comando do sistema operacional, assim
	como foi feito o bloqueio.\\
	
	Através desses bloqueios, é possível tornar as transações serializáveis, isto é, o resultado da operação de várias transações simultâneas é o mesmo obtido se elas fossem realizadas uma após a outra~\cite{kon94}. Um protocolo para a realização dessa serialização seria o protocolo de bloqueio de duas fases, onde na primeira fase ocorreria o bloqueio de todos os arquivos a serem usados nessa transação, e na segunda fase seria a liberação conjunta de todos os arquivos, após a realização das operações dentro dessas fases.
	Porém esse protocolo pode gerar um \textit{deadlock}, onde algum processo esperaria a liberação de um arquivo que foi bloqueado por outro processo, que também estaria esperando a liberação de um arquivo que foi bloqueado por aquele primeiro processo, por exemplo.
	
	%\item Segurança
	\subsubsection{Segurança}
	Os recursos devem ser protegidos contra acessos ilegais, permitindo somente a execução das operações solicitadas de um usuário conhecido. 
	Compartilhar arquivos entre vários ambientes e usuários é uma das vantagens que os sistemas de arquivos distribuídos nos trazem. Porém, deixar que outras pessoas possam acessar arquivos confidenciais, como, por exemplo, extrato de conta bancária, é um grande problema. Dessa forma, torna-se necessário adotar mecanismos de segurança, para evitar que pessoas desautorizadas tenham acesso aos arquivos do sistema. \\
	
	
	Sistemas como Unix adotam um método baseado em permissões para controlar o acesso aos seus arquivos. Cada arquivo possui informações sobre quais usuários podem acessá-lo e de que maneira.
	Nos sistemas distribuídos que rodam sob o Unix, quando um servidor recebe um pedido para enviar dados de um determinado arquivo, ele também recebe informações sobre qual usuário está tentando realizar tal acesso. Com isso, verifica se tal usuário tem permissão suficiente para realizar essa solicitação, fazendo uma comparação com as informações de permissões do arquivo.Além disso, os sistemas Unix possuem um usuário chamado \textit{root}, que possui permissão para acessar todos os arquivos da máquina local, de forma ilimitada. Isso funciona bem em sistemas locais, mas em uma rede já começam a surgir os problemas.
	Por exemplo, pode-se configurar as máquinas vizinhas de uma rede para confiarem no usuário \textit{root} entre elas. Assim, um usuário \textit{root} de uma máquina pode acessar os arquivos de outra máquina como se fosse o usuário \textit{root} local. O problema é se um usuário mal-intencionado consegue acesso como \textit{root}. Dessa forma ele teria acesso a todas as máquinas da rede. Outro problema é se o pedido que vier da rede for alterado para que o servidor acredite que quem está pedindo é ou o dono do arquivo ou o \textit{root}. Dessa forma, pode-se acessar qualquer arquivo da rede também. \\
	
	Outra forma de implementar esse controle de segurança é um sistema baseado em capacidades, que consiste de o cliente enviar ao servidor uma prova de que ele possui a capacidade de acessar um determinado arquivo. Na primeira vez que o usuário acessa tal arquivo, é enviado ao servidor sua identificação, e o servidor por sua vez retorna um código que é a sua prova de capacidade para acessar aquele arquivo. Nas próximas requisições, o cliente não precisa mais se identificar, bastando apenas enviar a prova de sua capacidade. Deve-se tomar cuidado para não criar provas de capacidade que sejam fáceis de serem forjadas. Até agora comentamos sobre o controle no acesso aos arquivos. Porém, caso haja outras máquinas no caminho de duas máquinas confiáveis, existe o risco de se ter dados interceptados ou, até mesmo, adulterados. Uma forma de se resolver esse problema é criptografar as informações antes de enviá-las.
	
	%\item Heterogeneidade
	\subsubsection{Heterogeneidade}
	No sistema distribuído que apresenta heterogeneidade o seu funcionamento não é afetado por diferenças em sistemas operacionais, protocolos de comunicação, linguagens de programação, formatos de dado e arquiteturas de hardware. 
	
	
	%\item Extensibilidade e abertura
	\subsubsection{Extensibilidade e abertura}
	Os interfaces devem ser claramente separados e publicamente disponíveis para facilitar o adicionamento de novos componentes e as extensões para componentes existentes.
	
	%\item Migração e balanceamento de carga
	\subsubsection{Migração e balanceamento de carga}
	Permitir a circulação de tarefas dentro de um sistema, sem afetar a operação do usuário ou aplicativos, e distribuir a carga entre os recursos disponíveis para melhorar o desempenho.
	
	
	
	%\end{itemize}
		
	\subsection{Nomes e localização}
	Todo arquivo armazenado em um sistema de arquivos possui um nome e um caminho, que o identifica unicamente em tal sistema.
	Um caminho representa um nó de uma estrutura de diretórios, que pode ser representada como uma árvore como mostra na Figura~\ref{fig:arv_dir}.
	Tal árvore possui uma raiz, e cada nó pode possuir mais árvores ou arquivos.
	Dessa forma, para localizar um arquivo em uma árvore de diretórios basta seguir o caminho do arquivo, e ao chegar no diretório final procurar pelo nome de tal arquivo.
	A forma como esse nome e esse caminho são definidos dependem muito do sistema operacional. Por exemplo, no Unix um caminho é definido como uma sequência de nomes de diretórios, todos separados pelo caractere ’/’. O último nome dessa sequência pode ser o nome do arquivo, ou de um diretório, caso se esteja definindo um caminho para o mesmo.
	Em sistemas distribuídos, é possível que se encontre o nome da máquina em que o arquivo se encontra dentro dessa definição de caminho. Porém procura-se deixar isso transparente para o usuário. Essa característica será detalhada mais adiante.
	
	\begin{figure}[htb]
		\begin{center}
			
			\includegraphics[clip,width=10.0cm]{images/image3.png}
			\caption{Árvore de diretórios}
			\label{fig:arv_dir}
		\end{center}
	\end{figure}
	
	\subsection{Serviço de Nomes}
	O serviço de nomes se preocupa em indicar a localização de um determinado arquivo, dado o seu nome ou caminho. Se a localização do arquivo estiver armazenada no nome dele, como por exemplo raiz/dir1/teste1, então esse serviço de nomes não provê transparência de localização. Para prover essa transparência, o nome ou caminho de um arquivo não deve ter indícios de sua localização física, e caso esse arquivo mude de lugar, ou tenha várias cópias, o seu nome ou caminho não precisará
	ser alterado para poder ser encontrado. Para isso, o serviço precisa oferecer ou resolução por nomes, ou resolução por localização, ou ambos.\\
	
	Resolução por nomes mapeia nomes de arquivos legíveis por humanos, normalmente \textit{strings}, para nomes de arquivos legíveis por computadores, que normalmente são números, facilmente manipuláveis pelas máquinas. Por exemplo, o endereço "www.endereco.com" ~ é mapeado para o IP 111.222.111.222. Através desse conjunto de números é possível encontrar uma máquina na rede IP, utilizando-se de tabelas de rotas de endereços mascarados que indicam como chegar à posição desejada.\\
	
	Resolução por localização mapeia nomes globais para uma determinada localização. Por exemplo, números de telefone, onde temos código do país, da localidade, etc. Se transparência por nome e por localização estiverem sendo utilizadas, pode ser muito difícil realizar um roteamento para determinar a localização de um determinado nome. Pode-se pensar em soluções com servidores
	centralizados ou distribuídos, porém os centralizados podem se tornar um gargalo, enquanto os distribuídos precisam usar alguma técnica de descentralização, como por exemplo cada servidor é responsável por um determinado subconjunto de arquivos, ou cada um cuidaria de resolver a localização de determinados tipos de arquivos, etc.
	
	%texto.... referência \cite{borsley96}
		



 

	


%\chapter{Desenvolvimento}
\chapter{Computação na Nuvem}
	

 

	


Este capítulo detalha o \textit{BFT-SMaRt} apresentando seu conceito, princípios de projeto, protocolos centrais, reconfiguração, implementação, configurações alternativas e finalizando nas conclusões do capítulo.


%	\section{BTF-SMART}
	\section{Conceitos}
	Na área da computação, a replicação de máquina de estado (SMR) é um método para implementação de um serviço tolerante a falha, onde um conjunto de servidores trabalham coordenadamente para prover determinado serviço para a(s) máquina(s) cliente(s). \\
		
	O \textit{BFT-SMaRt} é uma biblioteca de código aberto criada recentemente em linguagem Java. É composta por um conjunto de classes capaz de implementar um sistema distribuído robusto consistindo pela replicação das máquinas de estado tolerantes a falhas bizantinas. Falha bizantina ocorre quando uma máquina apresenta um comportamento arbitrário fora de sua especificação, por exemplo, quando um servidor é invadido e começa a rodar código malicioso. Além disso, o \textit{BFT-SMaRt} apresenta confiabilidade, modularidade, reconhecimento de sistema \textit{multicore}, suporte à reconfiguração e interface flexível de programação, como foi apresentado por Alchieri et al em \cite{bessani3}. \\
	
	Novamente em \cite{bessani3} é abordado como nos últimos anos a discussão sobre replicação de máquina de estado (\textit{State Machine Replication - SMR}) tolerantes a falhas bizantinas (\textit{Byzantine Fault-Tolerant - BFT}) tem se acirrado contendo pouco avanço prático, apenas protótipos usados para validar ideias apresentadas em artigos, assim dificultando a aplicação dessa prática em aplicações reais. Os autores acreditam que isto seja devido à falta de implementações de um SMR BFT robusto. O \textit{BFT-SMaRt} foi proposto tendo em mente contornar tal dificuldade, o qual almeja tanto alta performance em execuções livre de falhas quanto corretude mesmo com replicas que apresentam comportamento arbitrário. Incluindo o desenvolvimento de protocolos para transmissão de estados e reconfiguração. \\
	
	\section{Princípios do Projeto}
	
	%\begin{itemize}
		%\item Modelo Harmonioso de Falhas\\
		\subsection{Modelo Harmonioso de Falhas}
		\textit{BFT-SMaRt} tolera falhas bizantinas não-maliciosas por padrão. Num modelo de sistema realista mensagens podem ser enviadas, rejeitadas ou corrompidas enquanto processos podem executar de forma anormal sem que hajam terceiros envolvidos. Além disso, também é possível configurar o \textit{BFT-SMaRt} para que ele lide com falha bizantinas maliciosas, para tal ele provem assinaturas criptografadas. \\
		
		%\item Simplicidade\\
		\subsection{Simplicidade}
		A ênfase na corretude dos protocolos levou aos projetistas evitarem otimizações no código-fonte que poderiam acarretar em complexidade desnecessária tanto em tempo de desenvolvimento ou codificação. Este foi um dos motivos para a biblioteca ter sido desenvolvida em Java ou invés de outras linguagens de programação em alto nível, como C/C++ ou Python.\\
		
		%\item Modularidade\\
		\subsection{Modularidade}
		\textit{BFT-SMaRt} implementa um protocolo SMR modular que utiliza uma primitiva de consenso bem definida. Além de módulos responsáveis por garantir uma comunicação ponto-à-ponto confiável, ordenação de solicitações de clientes e o consenso entre SMRs, o \textit{BFT-SMaRt} também implementa módulos de transferência de estados e reconfiguração, os quais são totalmente separados do protocolo de \textit{agreement}.\\
		
		\begin{figure}[htb]
			\begin{center}
				
				\includegraphics[clip,width=13.0cm]{images/image4.png}
				\caption{A modularidade do BFT-SMaRt. Adaptado de Alchieri~\cite{bessani3}}
				\label{fig:vis_sis}
			\end{center}
		\end{figure}
		
		%\item  Interface de Programação de Aplicação Simples e Extensível\\
		\subsection{Interface de Programação de Aplicação Simples e Extensível}
		A biblioteca java encapsula toda a complexidade de uma replicação de máquinas de estado tolerante a falhas bizantinas (BFT SMR) dentro de uma API que pode ser utilizada para a implementação de serviços determinísticos. Utilizando métodos \textit{invoke}(comando) para enviar comandos às réplicas que implementam o método \textit{execute}(comando) cujo objetivo é processar o comando recebido. Entretanto, caso a aplicação necessite de comportamentos especializados não suportado por este modelo de programação é possível utilizar outras chamadas ou \textit{plug-ins} tanto no lado do cliente quanto do servidor.\\
		
		
		%\item Consciência de ambiente Multi-Core\\
		\subsection{Percepção de ambiente Multi-Core}
		\textit{BFT-SMaRt} é capaz de aproveitar a arquitetura \textit{multicore} dos servidores para diminuir o tempo de processamento em regiões críticas do protocolo.\\
		
		%\item Modelo do Sistema\\
		\subsection{Modelo do Sistema}
		\textit{BFT-SMaRt} assume o modelo usual para sistemas BFT SMR: são necessárias n $\geq$ 3f+1 réplicas para tolerar f falhas bizantinas. Entretanto, visto que o sistema suporta reconfiguração, é possível modificar n e f em tempo de execução. Além disso, o sistema permite ser configurado para utilizar apenas n $\geq$ 2f+1 réplicas para tolerar f falhas de sistema. Porém, independente da configuração, o sistema necessita de conexões ponto-a-ponto confiáveis entre os processos de comunicação. Essa conexão é realizada utilizando \textit{message authentication code} (MAC) sobre o protocolo TCP/IP.\\
	%\end{itemize}
	
	\section{Protocolos Centrais}
	
	%\begin{itemize}
		%\item Total Order Multicast\\
		\subsection{\textit{Total Order Multicast}}
		Em um sistema distribuído, um algoritmo de ordenamento total é um protocolo de mensagem \textit{broadcast} que garante a entrega das mensagens de forma confiável e na mesma ordem para todos os participantes.\\
		
		O ordenamento total \textit{multicast} é possível graças ao \textit{Mod-SMaRt}, um protocolo modular que implementa BFT SMR utilizando uma primitiva de consenso. Durante sua fase de execução normal, a qual ocorre na ausência de falhas e na presença de sincronismo entre as outras réplicas, clientes enviam suas solicitações para todas as réplicas e aguardam pela resposta. O ordenamento total é alcançado através de uma sequência de instâncias de consenso, cada uma delas decidindo sobre um lote de solicitações de clientes. Cada instância é composta por três passos de comunicação. O primeiro passo solicita que o líder do consenso envie uma mensagem de \textit{PROPOSE} para cada réplica. Esta etapa é seguida por duas etapas de mensagens de todos para todos compostas de mensagens \textit{WRITE} e \textit{ACCEPT}. Onde as mensagens de \textit{PROPOSE} contém o lote de solicitações, \textit{WRITE} e \textit{ACCEPT} contém o \textit{hash} criptografado do lote.\\
		
		Quando uma falha ocorre ou alguma replica encontra-se dessincronizada das demais, o \textit{Mod-SMaRt} pode mudar para a fase de sincronização. Durante esta fase um novo líder é eleito e as réplicas são forçadas a entrarem na mesma instância de consenso. Este “pulo” pode causar com que algumas réplicas ativem o protocolo de transferência de estados.\\
		
		%\item Transferência de Estado\\
		\subsection{Transferência de Estado}
		A fim de implementar uma SMR que possa ser usada na prática, faz-se necessário que as réplicas possam ser reparadas e reintegradas ao sistema sem que todo o sistema de replicação seja reiniciado. Para garantir tal característica, o \textit{BFT-SMaRt} implementa algumas ideias chaves: (1) armazenar o log dos lotes de operações em execução em apenas um disco, (2) tirar instantâneos de estados (\textit{snapshots}) em diferentes pontos da execução em várias réplicas e (3) realizar transferência de estados de forma colaborativa, cada réplica enviando diferentes partes do estado para a réplica que está sendo recuperada.\\
	%\end{itemize}
	
	\section{Reconfiguração}
	O \textit{BFT-SMaRt} provê um protocolo especial que permite a adição ou execução de réplicas em tempo de execução. Porém, tal processo só pode ser iniciado pelos administradores executando um cliente com permissão de gerenciamento (\textit{View Manager}), por motivos de segurança.\\
	
	\section{Implementação}
	\textit{BFT-SMaRt} foi desenvolvido contendo menos de treze mil e quinhentas linhas de código Java distribuídos em cerca de noventa arquivos. Tal característica é significativamente menor do que ocorre em sistemas similares que geralmente possuem mais de vinte mil linhas de código. \\
	
	Um ponto chave quando se está implementando um mediador (\textit{middleware}) de replicação de alta vazão é como separar as várias tarefas do protocolo em uma arquitetura eficiente e robusta. No caso de uma replicação de máquinas de estado tolerante a falhas bizantinas (BFT SMR) existem dois requisitos adicionais: o sistema precisa lidar com centenas de clientes e resistir a possíveis comportamentos maliciosos tanto por parte dos clientes quantos das outras replicas.\\
	
	A Figura \ref{fig:image5} apresenta a arquitetura central com as \textit{threads}  usadas para o processamento das mensagens arquitetadas pela implementação do protocolo. Nesta arquitetura, todas as \textit{threads} comunicam através de filas delimitadas. A figura também mostra qual \textit{thread} alimenta e consume informações de cada fila. \\  
	
	\begin{figure}[htb]
		\begin{center}
			
			\includegraphics[clip,scale=0.57]{images/image5.png}
			\caption{Processamento de mensagens arquitetadas entre replicas do \textit{BFT-SMaRt}. Adaptado de Alchieri~\cite{bessani3} }
			\label{fig:image5}
		\end{center}
	\end{figure}
	
	As solicitações dos clientes são recebidas através do estoque de \textit{thread} provida pelo \textit{Framework} de comunicação \textit{Netty}. Assim que uma mensagem proveniente do cliente é recebida, é verificado se trata-se de uma solicitação ordenada ou não ordenada. Solicitações não ordenadas, as quais são geralmente aplicadas por comandos de apenas leitura, são entregadas diretamente para o serviço de implementação. No caso de uma solicitação ordenada, elas são entregues para o gerenciador de clientes, o qual verifica a integridade da solicitação, caso esteja integra, a solicitação é encaminhada para a fila do respectivo cliente. Perceba que o endereço MAC dos clientes são verificados pelo \textit{Netty threads}, desta forma as máquinas \textit{multi-core} e multi-processadas vão naturalmente aproveitar de seu poder para conquistar uma alta vazão. \\ 
	
	A \textit{thread} proponente é responsável por juntar uma carga de solicitações e transmitir a mensagem \textit{PROPOSE} do protocolo de consenso. O BFT-SMaRt preenche carga com solicitações pendentes até que: (a) seu tamanho alcance o máximo definido no arquivo de configuração; ou (b) não haja mais solicitações sobrando para serem adicionadas. Esta \textit{thread} só está ativa na replica líder. \\
	
	Cada mensagem \textit{m} que deve ser enviada de uma réplica para outra é colocada na fila de saída pela qual uma \textit{thread} emissora vai serializar a mensagem \textit{m}, produzir o endereço MAC que será anexado na mensagem e, enfim, enviar utilizando \textit{sockets TCP}. Do ponto de vista da réplica que irá receber a mensagem, uma \textit{thread} receptora vai ler \textit{m}, autenticar (validar seu MAC), desserializar e colocar na fila de entrada, onde todas as mensagens recebidas de outras réplicas são armazenadas em ordem para serem processadas.  \\
	
	A \textit{thread} processadora de mensagens é responsável por processar as mensagens provenientes do protocolo \textit{BFT SMR}. Esta \textit{thread} carrega as mensagens da fila de entrada e as processam caso façam parte do consenso que está sendo executado, entretanto, caso a mensagem pertença a um consenso que ainda será executado, ela é processada posteriormente, quando seu consenso estiver ativo. Caso a mensagem não se encaixe em nenhum dos dois cados, ela é apenas descartada. \\
	
	 Quando um consenso chega ao fim em uma réplica, ele é marcado como decidido e a carga que o possui também é marcada como decidida e colocada na fila das cargas decididas. Então, a \textit{thread} de entrega é chamada para coletar as cargas que estejam armazenados nesta fila, desserializar todas as solicitações da carga, remover cada uma delas das filas de seus respectivos clientes e marcar o  consenso corrente como finalizado. Após isso, a \textit{thread} de entrega invoca o serviço de réplica  para executar a solicitação e gerar a reposta correspondente. Quando terminar de gerar a resposta, o serviço de réplica a adiciona na fila de resposta. A \textit{thread} de resposta carrega as respostas armazenadas nessa fila e as manda para seus referidos clientes. \\
	 
	 O cronômetro de pedido da \textit{thread} é ativado periodicamente afim de verificar se alguma solicitação permanece como não respondida por mais tempo do que o delimitado por um tempo de \textit{timeout} predefinido em alguma fila de solicitações. A primeira vez que este cronômetro expira para alguma solicitação, faz com que ela seja encaminhada para o líder corrente. A segunda vez que este cronômetro expira para a mesma solicitação, a instância atual do protocolo de consenso é paralisado e a fase de sincronização é ativada. A base lógica destes cronômetros é a seguinte: dada uma rede em condições normais, o \textit{timeout} pode ser causado por algum cliente que não enviou a solicitação ao líder ou pelo líder que não encaminhou os pedidos das solicitações dos clientes. Visto que tipicamente existem muitos clientes para poucos servidores, é esperado que ocorra mais falhas no lado dos clientes do que dos servidores, por isto o protocolo do \textit{BFT-SMaRt} assume que o erro ocorreu no lado do cliente, suspeita-se do líder somente se o problema persistir. \\
	 
	 \section{Configurações Alternativas}
	 
	 Como mencionado nas seções anteriores, por padrão o \textit{BFT-SMaRt} tolera falhas bizantinas não maliciosas. Entretanto, o sistema pode ser configurado para suportar dois outros modelos de falhas.\\
	 
	 %\begin{itemize}
	 	%\item \textit{Crash Fault Tolerance} \\
	 	\subsection{Falhas de Sistema}
	 	\textit{BFT-SMaRt} suporta uma configuração de um parâmetro que caso seja ativado faz com que o sistema tolere apenas falhas de sistema. Quando esta propriedade está ativa, o sistema tolera  f \textless  n/2 (minoria simples), o que implica alterações em todos os passos necessários do protocolo, inclusive ignorando o passo de \textit{WRITE} durante a execução do consenso. Fora essas alterações, os protocolos são os mesmos do caso de tolerância de falha bizantina. 
	 	
	 	%\item Falhas Bizantinas Maliciosas\\
	 	\subsection{Falhas Bizantinas Maliciosas}
	 	Trabalhos anteriores ao \textit{BFT-SMaRt} demonstraram que o uso de assinaturas com chaves públicas sobre solicitações tornam impossível para os clientes forjarem vetores MAC e forçar que o líder seja alterado. Por padrão, o \textit{BFT-SMaRt} não utiliza assinaturas de chave pública além da utilizada para estabelecer chaves simétricas compartilhadas entre réplicas e durante a mudança do líder. Contudo, o sistema opcionalmente permite o uso de solicitações assinadas para evitar esse problema.\\
	 	
	 	Os mesmos trabalhos também mostram que líderes maliciosos podem lançar ataques de degradação de performance indetectáveis, fazendo com que a vazão do sistema caia drasticamente. Até o presente momento, o \textit{BFT-SMaRt} não apresenta meios de defesa contra este tipo de ataque. \\
	 	
	 	Por fim, o fato do \textit{BFT-SMaRt} ter sido desenvolvido em Java faz com que seja fácil aplicar o sistema em diferentes plataformas. Esta escolha permitiu que o time de desenvolvimento evitasse falhas de um único nó causadas por eventos acidentais (por exemplo, algum \textit{bug} ou problemas de infraestrutura) ou ataques maliciosos aproveitando de vulnerabilidades comuns.  \\
	 %\end{itemize}
	   
	
	%texto.... referência \cite{borsley96}
		
		\section{Conclusões do Capítulo}
	Neste capítulo foi detalhado a biblioteca \textit{BFT-SMaRt} apresentando seu conceito, princípios de projeto, protocolos centrais, reconfiguração, implementação e configurações alternativas.	Estes conceitos são fundamentais para a construção de um serviço de metadados possuindo alta confiabilidade e disponibilidade. 
	
\input{chapters/capitulo4}
Este capítulo está dividido em duas seções nas quais serão apresentados os conceitos de RAID e as peculiaridades de suas principais categorias. A primeira seção do capítulo apresenta os conceitos de RAID e, posteriormente, divide-se em subseções para os níveis RAID 0, RAID 1, RAID 2, RAID 3, RAID 4, RAID 5, RAID 6 e RAID 50. A segunda seção finaliza com conclusões sobre o capítulo

%	\section{RAID}
\section{Conceitos}
O desempenho da unidade de processamento e da memória principal cresceu rapidamente. Entre os anos de 1974 e 1984, esse desempenho aumentou cerca de 40\% ao ano. Deste modo uma instrução de CPU que realiza uma operação de entrada ou saída para acessar um disco tem a possibilidade de se tornar um grande gargalo, devido ao fato de que o dispositivo de armazenamento magnético, em questão de velocidade de acesso, não acompanha o crescimento dos dois componentes principais, o processador e a memória.\\

O termo RAID é originalmente a sigla de \textit{redundant array of inexpensive disk}(vetor redundante de discos econômicos), no entanto, atualmente é mais conhecido como \textit{redundant array of independent disk}(vetor redundante de discos independentes). O RAID se baseia no uso de discos extras para aumentar o desempenho do processo de leitura e escrita da unidade de armazenamento, ou para recuperar a informação original em caso de uma falha num disco. Patterson, um dos desenvolvedores da tecnologia RAID, demonstrou que sem um controle de tolerância à falhas grandes vetores de discos econômicos não podem ser considerados úteis devido a sua baixa taxa de confiabilidade~\cite{patterson88}. \\


A proposta inicial de Patterson ao desenvolver o RAID era de que, para superar o obstáculo da falta de confiabilidade era necessário o uso de discos extras que possuíssem informação redundante que possibilitassem a recuperação total das informações originais no caso da falha de algum disco. Vale ressaltar que a proposta inicial do RAID foi desenvolvida utilizando o \textit{hardware} como referencia. No entanto, em seu artigo original~\cite{patterson88}, Patterson frisa que as ideias centrais do projeto poderiam ser aplicadas facilmente na implementação de \textit{software}. Este ponto é de fundamental importância para este trabalho. A ideia central apresentada por Patterson era de quebrar os vetores em grupos confiáveis, onde cada grupo possua discos extras, os quais possuem informação redundante que seriam utilizados para manter a confiabilidade dos grupos. Dessa forma, quando um disco falhar, o disco em falha será substituído por um novo disco em um espaço curto de tempo,  e a informação que estava contida no antigo disco será totalmente reconstruída no novo disco utilizando as informações redundantes contidas no vetor. O tempo de espera ficou conhecido como \textit{mean time to repair (MTTR)} ou, em uma tradução livre, "tempo de conserto". O MTTR pode ser reduzido se o sistema possuir discos extras que funcionem como peças sobressalentes em estado de prontidão, de forma que, quando um disco falha, ele é trocado por um desses discos extras de forma eletrônica, sem que haja a necessidade de intervenção humana (bastando apenas que, de tempos em tempos, um profissional técnico troque os discos defeituosos por novos discos). Patterson, ao desenvolver o conceito de RAID, assumiu que as falhas de discos são independentes e exponenciais. \\

O RAID é separado em diferentes níveis, sendo que cada nível possuiu seus objetivos distintos e características intrínsecas. Os principais níveis de RAID serão apresentados e explicados nas próximas subseções deste trabalho.
\\

\subsection{RAID 0}

Também conhecido como fracionamento, pois os dados são fracionados igualmente entre dois ou mais discos, sem a preocupação com informação de paridade, tolerância a falhas ou redundância. Devido ao fato da completa falta de tolerância a falhas no RAID 0, a perda de um disco acarreta na perda (sem qualquer chance de recuperação) de todos os dados contidos no disco e da inutilização de todas as outras frações dos dados que estavam armazenadas nos outros discos do vetor, visto que estes arquivos jamais serão completamente recuperados.
\\

Por esta razão, o RAID 0 é usado apenas nos casos em que deseje aumentar o desempenho do sistema, ou como forma de criar uma grande unidade lógica de armazenamento que utilize dois ou vários discos físicos. Inclusive, é possível modelar um sistema em RAID 0 que utilize discos de diferentes capacidades. Entretanto, é importante frisar que os discos com maior quantidade de armazenamento disponível serão limitados ao espaço máximo do menor disco do vetor. Por exemplo, em um vetor montado com dois discos, um com 500GB e um segundo de 280GB, possuiria o total de 560GB de armazenamento, ou seja, 280 X 2. Nitidamente o disco de 500GB está sendo subutilizado neste vetor. 
\\

A Figura ~\ref{fig:raid0} representa o diagrama de um vetor modelado com o RAID 0 utilizando-se dois discos. Como os dados contidos nos discos 0 e 1 podem ser recuperados simultaneamente, isso resulta na ilusão de que o \textit{drive} é mais rápido.\\

\begin{figure}[htb]
	\begin{center}
		
		\includegraphics[clip,scale=0.5]{images/RAID_0.png}
		\caption{Diagrama do RAID 0. }
		\label{fig:raid0}
	\end{center}
\end{figure} 

\subsubsection{Desempenho}

RAID 0 pode ser usado em áreas onde o desempenho é crucial, considerando que a integridade dos dados seja de pouca relevância, como ocorre no caso de jogos \textit{online}.\\

\subsection{RAID 1}

RAID 1 é também conhecido como espelhamento, pois este nível consiste de uma cópia exata de um conjunto de dados espalhado por dois ou mais discos. A configuração do RAID 1 não proporciona paridade ou fracionamento de dados, visto que os dados são apenas espelhados dentro de todos os discos pertencentes do vetor, sendo que o espaço de armazenamento do vetor não pode ser maior do que o disco membro com o menor espaço de armazenamento. O vetor continuará funcional enquanto ao menos um membro do vetor esteja disponível. Esta configuração é útil para sistemas onde a confiança e o desempenho de leitura seja mais importante do que performasse de escrita ou um armazenamento eficiente. \\

A Figura ~\ref{fig:raid1} representa o diagrama de um vetor modelado com o RAID 1 utilizando-se dois discos.\\

\begin{figure}[htb]
	\begin{center}
		
		\includegraphics[clip,scale=0.5]{images/RAID_1.png}
		\caption{Diagrama do RAID 1. }
		\label{fig:raid1}
	\end{center}
\end{figure} 

\subsubsection{Desempenho}

Qualquer solicitação pode ser atendida por qualquer disco do vetor. Sendo que no caso de solicitações de escrita o desempenho é comprometido pois é necessário que ela seja replicada em todos os discos membros do vetor. \\

\subsection{RAID 2}

A configuração RAID 2 fraciona o dado ao nível do \textit{bit} ao invés de blocos, além de utilizar a técnica do Código de Hamming para a correção de erros. O eixo de rotação de todos os discos são sincronizados e o dado é fracionado de forma que cada sequência de \textit{bits} sejam armazenadas em discos diferentes. A paridade do código de hamming é calculada sobre os \textit{bits} correspondentes e armazenadas em ao menos um disco de paridade.\\

A Figura ~\ref{fig:raid2} representa o diagrama de um vetor modelado com o RAID 2 utilizando-se sete discos, sendo três deles utilizados como discos de paridade.\\

\begin{figure}[htb]
	\begin{center}
		
		\includegraphics[clip,scale=0.35]{images/RAID_2.png}
		\caption{Diagrama do RAID 2. }
		\label{fig:raid2}
	\end{center}
\end{figure} 

\subsubsection{Desempenho}
Por utilizar o código de Hamming, possui forte tolerância a falhas. Entretanto, como os discos rígidos atualmente possuem correção interna de erros, o RAID 2 acabou perdendo seu propósito. 

\subsection{RAID 3}

Diferente do RAID 2, o RAID 3 fraciona os dados ao nível de \textit{byte}, tendo um disco dedicado de paridade. O eixo de rotação dos discos são sincronizados e o dado é fracionado de forma que cada sequência de \textit{bytes} sejam armazenadas em discos diferentes. A paridade é calculada sobre os \textit{bytes} correspondentes e armazenadas no disco de paridade. Apesar de existir casos de implementação, o RAID 3 dificilmente é utilizado na prática devida a forma de fracionar os dados, onde a fração ao nível de \textit{bytes} é menos eficiente que a em nível de blocos, que será abordada posteriormente.\\

A Figura ~\ref{fig:raid3} representa o diagrama de um vetor modelado com o RAID 3 utilizando-se quatro discos, sendo um deles utilizado como disco de paridade.\\

\begin{figure}[htb]
	\begin{center}
		
		\includegraphics[clip,scale=0.5]{images/RAID_3.png}
		\caption{Diagrama do RAID 3. }
		\label{fig:raid3}
	\end{center}
\end{figure} 

\subsubsection{Desempenho}
Esta configuração se encaixa para aplicações que necessitam de altas taxas de transferência de grandes sequências de dados sequenciais, como por exemplo, edição de vídeos não comprimidos. Todavia, aplicações que fazem uso de leituras e escritas em localizações aleatórias do disco tendem a ter um desempenho inferior, visto que as frações de um mesmo dado estão espalhadas nas mesmas seções dos vários discos do vetor.\\

\subsection{RAID 4}
Diferente dos RAID 2 e 3, o RAID 4 fraciona os dados ao nível do bloco de dados, tendo um disco do vetor dedicado a paridade. \\

A Figura ~\ref{fig:raid4} representa o diagrama de um vetor modelado com o RAID 4 utilizando-se quatro discos, sendo um deles utilizado como disco de paridade.\\

\begin{figure}[htb]
	\begin{center}
		
		\includegraphics[clip,scale=0.5]{images/RAID_4.png}
		\caption{Diagrama do RAID 4. }
		\label{fig:raid4}
	\end{center}
\end{figure} 

\subsubsection{Desempenho}
Sempre que os dados são escritos no vetor, o novo dado sobre paridade deve ser recalculado e escrito para o respectivo disco de paridade antes que qualquer requisição de escrita seja realizada. Por causa dessas operações de leitura e escrita, o disco de paridade é o fator limitante do desempenho total do vetor.\\

\subsection{RAID 5}
Similar ao RAID 4, o RAID 5 fraciona os dados ao nível do bloco de dados, porém tendo as informações de paridade espalhadas entre os discos do vetor. Isto exige que todos os discos (exceto um) tenham os dados. No caso da falha de apenas um disco, leituras subsequentes podem ser calculadas utilizando-se os dados restantes e das informações de paridade para recuperar o bloco faltoso. Entretanto, caso o disco que está indisponível é o portador dos dados de paridade, não acarreta na necessidade de nenhum cálculo a mais. É importante frisar que por suas características, o RAID 5 necessita de ao menos três discos.  \\

A Figura ~\ref{fig:raid5} representa o diagrama de um vetor modelado com o RAID 5 utilizando-se quatro discos.\\

\begin{figure}[htb]
	\begin{center}
		
		\includegraphics[clip,scale=0.5]{images/RAID_5.png}
		\caption{Diagrama do RAID 5. }
		\label{fig:raid5}
	\end{center}
\end{figure} 

\subsubsection{Desempenho}
O RAID 5 sofre pelo tempo demasiadamente grande necessário para reconstruir um arquivo a partir dos blocos restantes e do bloco de paridade, além da chance de mais de um disco ficar indisponível durante a fase de recuperação. Como a recuperação de um bloco de dados requer a leitura de todos os blocos de dados, abre-se a chance de perder todos os dados do vetor caso ocorra a perda de um segundo disco.\\

\subsection{RAID 6}
Similar ao RAID 5, o RAID 6 fraciona os dados em nível do bloco de dados, porém tendo as informações de paridade espalhadas entre os discos do vetor e duplicadas. A novidade da paridade dupla garante tolerância a falhas até o caso da perda de dois discos. RAID 6 exige a utilização de no mínimo quatro discos. \\ 

Semelhante ao caso do RAID 5, a queda de um disco acarreta na redução do desempenho de todo o vetor até que o disco defeituoso tenha sido substituído.\\

A Figura ~\ref{fig:raid6} representa o diagrama de um vetor modelado com o RAID 6 utilizando-se cinco discos.\\

\begin{figure}[htb]
	\begin{center}
		
		\includegraphics[clip,scale=0.5]{images/RAID_6.png}
		\caption{Diagrama do RAID 6. }
		\label{fig:raid6}
	\end{center}
\end{figure} 

\subsubsection{Desempenho}
RAID 6 não possui penalidade de desempenho sobre operações de leitura, contudo, possui penalidade de desempenho nas operações de escrita devido ao \textit{overhead} associado aos cálculos de paridade.\\

\subsection{RAID 50}
Existe o termo originalmente conhecido como RAID híbrido, o qual é definido pela aplicação de um nível de RAID sobre outro vetor formado por outro nível de RAID. O último vetor gerado é conhecido como o vetor do topo. Logo, no caso do RAID 50 (ou RAID 5+0) é uma combinação híbrida que usa as técnicas de RAID 5 com paridade em conjunção com a segmentação de dados do RAID 0. Em outras palavras, o RAID 50 não passa da aplicação do RAID 0 sobre um vetor de RAID 5. \\

A Figura ~\ref{fig:raid50} representa o diagrama de um vetor modelado com o RAID 50 utilizando-se três conjuntos de quatro discos cada.\\

\begin{figure}[htb]
	\begin{center}
		
		%\includegraphics[clip,width=15.0cm]{images/RAID_50.png}
		\includegraphics[clip,scale=0.2]{images/RAID_50.png}
		\caption{Diagrama do RAID 50. }
		\label{fig:raid50}
	\end{center}
\end{figure} 

\subsubsection{Desempenho}
Possui alta taxa de transferência, porém com alto custo de manutenção e expansão.

%texto.... referência~\citep{patterson88}

\section{Conclusões do Capítulo}
Neste capítulo foi apresentado os conceitos básicos do que é RAID, suas características positivas e negativas, além de abordar sobre as topologias mais importantes do RAID, ou seja, RAID 0, RAID 1, RAID 2, RAID 3, RAID 4, RAID 5, RAID 6 e RAID 50. Vale destacar que ainda existem outros níveis de RAID oriundos da combinação destes níveis básicos apresentados.









\chapter{Proposta de Sistema de Arquivos Distribuído}
	\section{Metadados} 
	
	Os sistemas de arquivos, como ambientes propícios para a recuperação de informações, têm na utilização de metadados a padronização das formas de representação e a possibilidade de garantia de interoperabilidade entre sistemas, favorecendo a integridade e a acessibilidade dos recursos informacionais de forma eficiente pelo usuário final. Os metadados são "dodos de dodos", ou seja, as informações essenciais sobre os arquivos armazenados, indispensáveis para fazer o gerenciamento de arquivos, indicando as características inerentes deles. A Tabela~\ref{tab:metadado} mostra os atributos encontrados em um metadado.
	
	\capstartfalse
	\begin{table} [htb]
		\caption{Elementos de metadadso}
		\centering
		\begin{tabular}{|l|l|} \hline
			\textbf{Informação} & \textbf{Descrição} \\ \hline
			
			Nome do arquivo		& O nome do arquivo, incluindo o caminho do diretório.\\ \hline
			Proprietário		& O identificador de usuário que é o dono do arquivo.\\ \hline
			Data de criação     & Data que o arquivo foi criado.\\ \hline
			Data de acesso		& Data de último acesso do arquivo. \\ \hline
			Data de atualização	& Data de última atualização feito sobre o arquivo. \\ \hline
			Tamanho				& Espaço ocupado pelo arquivo ao longo do disco. \\ \hline
			Tipo de arquivo		& indica se ele é uma pasta ou um arquivo normal.  \\ \hline
			Modo de acesso		& indica a permissão para acessar o arquivo. \\ \hline
			Bloqueio			& indica se o arquivo está bloqueado para acesso. \\ \hline
			Lista de servidores	& lista de todos os servidores que armazena partes do arquivo. \\ \hline
			
		\end{tabular}
		\label{tab:metadado}
	\end{table}
	\capstarttrue

	\section{Arquitetura do sistema}
	
	A arquitetura do SAD proposto neste artigo é composto por vários clientes e servidores distribuídos, onde cada um deles está conectado através de internet, como a Figura~\ref{fig:vis_sis} mostra resumidamente. Neste sistema, cada servidor se comporta como um disco rígido em sistema de RAID, armazenando distribuidamente as partes dos arquivos e paridades associadas.
	
	
	%O nosso objeivo neste artigo é a construção de um SAD baseado na computação em nuvem que possui menor \textit{overhead} de espaço, comparado a 200\% de método de replicação, sem degradar muito o performance e segurança de dados. 
	%Para isso, nós observamos em técnica de redundância usado na tecnologia RAID, onde consegue diminuir o espaço ocupado pela parte redundante com atribuição de paridade nos arquivos, ao invés de simples replicação. Na mesma tecnologia também foi 

	
	
	\begin{figure}[htb]
		\begin{center}
			
			\includegraphics[clip,width=10.0cm]{images/image1.png}
			\caption{Visão geral do sistema}
			\label{fig:vis_sis}
		\end{center}
	\end{figure}
	
	
	
	\subsection{Servidor}
	
	Grupo de servidores que fazem armazenamento e gerenciamento de dados. \\
	
	Em um sistema de arquivos local, normalmente o metadado e o conteúdo de um arquivo são armazenados na mesma unidade de armazenamento. No caso de um SAD, como um arquivo pode ser armazenados distribuidamente entre servidores diferentes, os metadados são espalhados por vários servidores, tendo assim, a necessidade de fazer a busca para acessar em um determinado metadado. Obviamente essa busca consome recursos operacionais, relativamente alta por tratar de uma transmissão na rede.   
	\begin{figure}[htb]
		\begin{center}
			
			\includegraphics[clip,width=10.0cm]{images/image7.png}
			\caption{\textit{Name node} e \textit{data node}}
			\label{fig:namenode}
		\end{center}
	\end{figure}
	
	\subsubsection{\textit{Name node}}
	
	O \textit{name node} é o conjunto de servidores que gerencia as informações dos arquivos armazenados, em forma de metadado.
	Mantém na sua memória local a árvore de diretórios, o que indica localização lógica dos arquivos. Quando recebe uma solicitação de cliente por um arquivo, pesquisa a sua árvore de diretórios para identificar qual diretório que este arquivo se encontra. Como resultado da pesquisa obtém o metadado do arquivo referente, conseguindo descobrir a sua localização física, a lista dos todos os \textit{data nodes} que possuem as partes do arquivo. Assim, também gerencia as comunicações entre clientes e \textit{data nodes}, informando os clientes quais são os servidores que devem estabelecer a conexão para enviar ou receber um arquivo. 
	\\
	
	Por sua propriedade como um interface entre clientes e \textit{data nodes}, a ocorrência de alguma falha na operação ou indisponibilidade causada pela queda do servidor tem  enorme influencia na execução do sistema, podendo até resultar em parada total do sistema. Assim, é muito importante elaborar uma esquema para manter o \textit{name node} protegidos contra falhas ou queda total. Em nosso sistema usaremos a biblioteca BFT-SMaRt, que fornece tolerância a falha no serviço através de replicação por máquina de estado. 
	
	\subsubsection{\textit{Data node}}
	
	Parte que realiza o armazenamento de arquivos em si.
	
	\subsection{Cliente}
	Programa cliente que é executado no computador de um usuário. 
	\\
	
	O programa cliente é capaz de dividir um arquivo em vários pedaços e gerar paridades. Por consequência, também é capaz de reconstruir um arquivo a partir de partes divididos. A Figura~\ref{fig:img6} mostra o procedimento.
	\begin{figure}[htb]
		\begin{center}
			
			\includegraphics[clip,width=15.0cm]{images/image6.png}
			\caption{Geração de paridade}
			\label{fig:img6}
		\end{center}
	\end{figure}
	
	Para assegurar a disponibilidade do sistema, é interessante que o usuário tenha acesso ao um arquivo mesmo que algum dos seus fragmentos de dado esteja indisponível por uma razão qualquer, recuperando o arquivo íntegro a partir de fragmentos disponíveis neste momento.\\
	
	Quando um cliente percebe que o arquivo que solicitou apresenta  
	A Figura~\ref{fig:img2} mostra esquema para recuperar um arquivo no lado do cliente.
	
	\begin{figure}[htb]
		\begin{center}
			
			\includegraphics[clip,width=10.0cm]{images/image2.png}
			\caption{Recuperando um arquivo de uma falha}
			\label{fig:img2}
		\end{center}
	\end{figure}
	
	\section{Operações}
	
	Nesta seção serão apresentadas as operações básicas que o SAD executa. Estas operações podem ser divididos em dois grupos dependendo dos entidades envolvidos na operação executada. Um deles é entre cliente e servidor, onde consiste das operações que podem ser encontradas em sistemas de arquivos em geral, focalizadas em fornecer um serviço de armazenamento de dados. Outro tipo é as operações que ocorrem entre servidores, voltadas para gerenciamento do serviço. Este tipo de operação não pode ser percebido por usuário, mantendo algumas transparências.
	
	
	%Os protocolos neste sistema indicam os tipos de mensagens que vai ser trocado entre cliente/servidor ou servidor/servidor. 
	
	\subsection{Cliente-Servidor}
	
	É um conjunto de operações semelhantes a que são implementadas em um sistema de arquivos local, aquele que é integrado na maioria dos sistemas operacionais comuns. Composto pelas operações sobre arquivos e pastas e suas execuções podem ocorrer entre cliente e \textit{name node} ou cliente e \textit{data node}.
	
	\subsubsection{Criar arquivos}
	
	Quando um cliente deseja criar um arquivo dentro de um SAD, primeiramente deve enviar para servidor do tipo \textit{name node} o metadado referente ao arquivo, o que contem as informações essenciais como nome, diretório, tamanho e entre outros. No lado do servidor, recebendo o metadado do arquivo a ser criado, extrai dele o nome e diretório do arquivo. 
	
	\subsubsection{Abrir arquivos}
	
	Abrir um arquivo existente provavelmente vai ser uma das tarefas mais usado no sistema. Geralmente é composto por três procedimentos básicos. Esta operação inicia pegando como referencia o nome e o diretório do arquivo, e procura por nome neste diretório. É muito importante que a busca pelo arquivo tenha uma boa eficiência, pois uma vez que este procedimento é executado muitas vezes e normalmente um servidor de grande porte possui  diretórios com milhares entradas encontrados dentro deles. Assim, a escolha do estrutura de dados para diretórios pode influenciar criticamente no desempenho do sistema. Depois de encontrar o arquivo desejado, será verificado algumas condições para decidir se o arquivo pode ser aberto mesmo. Uma condição é se o arquivo não esteja bloqueado por outros usuários, impedindo que seja aberto no momento. Outra é se o usuário que solicitou o acesso por este arquivo possui a permissão para fazer isso. Passando por checagem das permissões de acesso, finalmente o servidor envia uma resposta para cliente, indicando os servidores que armazenam cada pedaços do arquivo.
	 
	\subsubsection{Deletar arquivos}
	Deletar um arquivo vai seguir procedimentos semelhantes a operação de abrir arquivo. Em primeiro lugar recebe o nome e diretório do arquivo a ser apagado. Faz a busca pela árvore de diretórios para achar o diretório requisitado, e chegando em diretório correto, procura o arquivo por nome. Quando encontra o arquivo desejado, verifica se está satisfazendo as condições para efetuar a operação, checando a permissão de acesso e estado de bloqueio do arquivo.  
	
	\subsubsection{Fechar um arquivo aberto}
	
	Fechar o arquivo aberto por um cliente.
	
	\subsubsection{Editar um arquivo}
	
	Fazer alteração no arquivo armazenado.
	
	\subsubsection{Modificar metadado}
	
	Permite fazer alterações sobre as informações do arquivo, editando diretamente o metadado associado.
	
	
	\subsection{Servidor-Servidor}
	São operações ocorridos entre \textit{name node} e \textit{data node}, com função de gerenciamento de SAD.
	
	\subsubsection{Receber arquivo criado}
	
	Na operação de criar arquivo, o \textit{name node} define quais \textit{data nodes} vão ser utilizados para armazenar os fragmentos de arquivo a ser criado, calculando a melhor forma para distribuir igualmente a carga entre os \textit{data nodes}. Dessa forma, além de informar o cliente quais são os servidores que deve enviar os dados, deve avisar os \textit{data nodes} que vai chegar dados enviados pelo cliente.
	
	\subsubsection{Apagar arquivo deletado}
	
	Deleta os dados pertencentes ao arquivo solicitado para exclusão.
	
	\subsubsection{Transferir dados entre servidores}
	
	Transferência de arquivos entre \textit{data nodes}.
	
	
	
	

\section{Implementação}
Nesta seção serão apresentados os detalhes sobre a implementação do nosso sistema, incluindo informações sobre a programação e algoritmos utilizados. Todo o \textit{software} foi desenvolvido na linguagem Java com auxilio de várias bibliotecas padrões e do \textit{BFT-SMaRt}, o qual foi apresentado com mais detalhes no capítulo 4. 
\\


\subsection{Árvode de Diretórios}
Nesta seção será apresentado o pacote responsável por gerenciar todas as estruturas de diretório do sistema, junto com as classes contidas nele.
\\

\begin{itemize}
	\item \textbf{dt}
	\begin{itemize}
		\item DirectoryNode;
		\item DirectoryTree;
		\item LockList;
		\item LockType;
		\item Metadata.
	\end{itemize}
	\item \textbf{dt.directory}
	\begin{itemize}
		\item DirEntries;
		\item Directory.
	\end{itemize}
	\item \textbf{dt.file}
	\begin{itemize}
		\item Block;
		\item BlockInfo;
		\item BlockInfoList;
		\item FileDFS.
	\end{itemize}
\end{itemize}

As classes contidas no pacote \textit{dt} lidam com todas características e operações refentes aos diretórios. \textbf{LockType} e \textbf{LockList} são utilizadas para gerenciar o controle de acesso aos diretórios, sendo a primeira responsável por enumerar os tipos dos estados de acesso dos diretórios, enquanto a segunda gerência uma lista, na forma de \textit{ArrayList}, dos arquivos ou diretório que estão abertos por usuários.
Esta lista serve para que os programas clientes requisitarem ao servidores de metadado a atualização do estado de acesso. 
\\

A classe \textbf{Metadata} contêm e manipula as informações de metadado dos arquivos. 
Como mostra no Código~\ref{code:metadata}, implementa a interface \textit{Serializable} para poder transformar o conteúdo dos campos em sequência de \textit{bytes} e posteriormente ser enviado através da rede.

\begin{lstlisting}[basicstyle=\ttfamily\footnotesize, frame=single, caption=Classe Metadata, label=code:metadata]	
public class Metadata implements Serializable {
	private long creationTimeL;
	private long lastAccessTimeL;
	private long lastModifiedTimeL;
	
	private long size;
	
	private int lock;
	
			...
\end{lstlisting}	

Por fim, \textbf{DirectoryNode} e \textbf{DirectoryTree} compõem a árvore de diretórios do serviço de metadados, a primeira referente a cada nó da árvore e a segunda sobre a árvore em si, carregando o diretório raiz e os métodos das operações relacionadas.
\\

Dentro do pacote \textit{dt}, existem dois outros pacotes.
Primeiro é o \textit{directory}, responsável por tratar os diretórios em si, composto pelas classes \textbf{Directory} e \textbf{DirEntries}.
A classe \textit{Directory} trata dos diretórios em si composto por vários métodos de operações sobre pastas e possui no seu campo dois \textit{HashMaps}, um para armazenar as informações das pastas contidas e outro para arquivos contidos no diretório referenciado.
Esta classe, por ser um dos tipos de nó da árvore de diretórios, herda a classe \textit{DirectoryNode}.
O Código~\ref{code:directory} mostra as partes da implementação citadas.
\begin{lstlisting}[basicstyle=\ttfamily\footnotesize, frame=single, caption=Declaração e os campos da classe Directory, label=code:directory]
public class Directory extends DirectoryNode {
	private HashMap<String, Directory> dirs;
	private HashMap<String, FileDFS>   files;
	
			...
\end{lstlisting}

O \textit{DirEntries} é usado por serviço de metadados para informar aos clientes as entradas de um determinado diretório quando este foi aberto pelo usuário, na forma de uma lista de nomes dos diretórios e arquivos contidos nele. 
\\

Por fim, o segundo pacote dentro de \textit{dt} é nomeado de \textit{file}, o qual compõem as classes responsáveis para gerenciamento das informações sobre arquivos ou dos dados armazenados na forma de bloco. 
O \textbf{FileDFS} é a classe principal deste pacote, encarregando a parte das informações sobre arquivos.
Ele mesmo não possui muita funcionalidade, apenas carregando a classe \textit{BlockInfoList} no seu campo, como mostra no Código~\ref{code:filedfs}.
Por ser um tipo de nó da árvore de diretórios, herda as características da classe \textit{DirectoryNode}.
\begin{lstlisting}[basicstyle=\ttfamily\footnotesize, frame=single, caption=Classe FileDFS, label=code:filedfs]
public class FileDFS extends DirectoryNode {
	private BlockInfoList blockList;
	
	public FileDFS(String name, Directory parent, Metadata metadata, BlockInfoList blockList) {
		super(name, parent, metadata);	
		this.blockList = blockList;
	}
	
	public BlockInfoList getBlockList() {
		return blockList;
	}
}
\end{lstlisting}

Como mostra no Código~\ref{code:blockinfolist}, o \textbf{BlockInfoList} é a classe que possui as propriedades do arquivo, como a lista de \textit{BlockInfo}, o tamanho dos blocos, o tipo de RAID utilizado e o número dos servidores que armazenam os blocos.
A sua função principal é para que o serviço de metadados consiga informar aos clientes as informações operativas de um determinado arquivo, quando precisa enviar ou receber os blocos de dados deste arquivo.
\begin{lstlisting}[basicstyle=\ttfamily\footnotesize, frame=single, caption=Declaração e os campos da classe BlockInfoList, label=code:blockinfolist]
public class BlockInfoList implements Serializable {
	private ArrayList<BlockInfo> blocks;
	private long blockSize;
	private int  raidType;
	private int  nServers;
 
			...
\end{lstlisting}

Cada \textbf{BlockInfo} gerencia a localização de um bloco, indicando o endereço do servidor onde está armazenado e o identificador do bloco referente.
\\


A classe \textbf{Block} está encarregado da parte de conteúdo dos arquivos, gerenciando em forma de blocos.
Ela é usado para fazer a transferência de dados de arquivo, contidos em blocos, entre cliente e servidor de armazenamento.
Também tem papel de informar os servidores o identificador de cada bloco na hora de carregar ou armazenar o conteúdo no disco.
%\textbf{Block, BlockInfo e BlockInfoList} referentes aos blocos de dados nos quais os arquivos devem ser divididos antes de serem enviados pelo cliente ao serviço de armazenamento. 

%Dentre as classes previamente sitadas,\textit{Block} pode ser considerada como a principal, visto que é nela onde ficam armazenados os \textit{bytes} de dados, o identificador e os métodos necessários para recuperar os dados ou quebrar um dado arquivo em blocos. 

%Enquanto que \textit{BlockInfo} funciona como a ligação do bloco ao servidor de armazenamento par ao qual ele deve ser enviado, contendo as informações necessárias para que o cliente saiba para quem o bloco deve ser enviado. Por fim, \textit{BlockInfoList} possui os meios para gerenciar um \textit{ArrayList} dos \textit{BlockInfo} de um arquivo, além de manter o tamanho de cada bloco da lista, o número de servidores de armazenamento e o tipo de RAID em que os blocos foram criados.


\subsection{Serviço de Metadados}
Nesta seção a implementação do Serviço de Metadados será detalhada. O serviço supracitado foi desenvolvido em quatro classes principais e um auxiliar, as quais são listadas a seguir de acordo com  a estrutura do pacote.
\\

\begin{itemize}
	
	\item server
	\begin{itemize}
		\item ServerList;
	\end{itemize}
	\item server.meta
	\begin{itemize}
		\item RaidType;
		\item ServerConsole;
		\item ServerInfo;
		\item ServerMeta.
	\end{itemize}

\end{itemize}

A classe \textbf{RaidType} é apenas enumeração dos tipos de RAID em forma de campos constantes de tipo inteiro.
\\

A classe \textbf{ServerConsole} serve apenas como uma interface de inicialização do serviço de metadados, no qual o \textit{ServerConsole} verifica os parâmetros de inicialização, caso tenha algo errado é apresentado uma mensagem de erro e o programa finalizado. Em contra partida, se tudo estiver correto com os parâmetros, a classe \textit{ServerMeta} é instanciada.
Código~\ref{code:meta_con}.
\\

\begin{lstlisting}[basicstyle=\ttfamily\footnotesize, frame=single, caption=Classe ServerConsole, label=code:meta_con]	
public class ServerConsole {
	
	public static void main(String[] args){
		if(args.length < 3) {
			System.out.println("Use: java ServerConsole <processId> <raidType> <nServers> <verbose>");
			System.exit(-1);
		}
		boolean verbose = false;
		boolean test    = true;
		if(args.length > 3) {
			if(args[3].contains("v"))
			verbose = true;
			if(args[3].contains("n"))
			test = false;
		}
		
		new ServerMeta(Integer.parseInt(args[0]), Integer.parseInt(args[1]), Integer.parseInt(args[2]), verbose, test);
	}
}
\end{lstlisting}	

A classe \textbf{ServerList} implementa a lista de servidores de armazenamento, em forma de \textit{ArrayList} do Java, e suas operações relacionadas.
\\

Cada item da lista é de classe \textbf{ServerItem}, que possui as informações sobre servidores de armazenamento, como mostra no Código~\ref{code:serv_info}.

\begin{lstlisting}[basicstyle=\ttfamily\footnotesize, frame=single, caption=Campos da classe ServerInfo, label=code:serv_info]	
public class ServerInfo {
	private String hostName;
	private int    port;
	private long   capacity;
	private long   size;
	private long   lastID;
	private long   lastAccessTime;

			...
\end{lstlisting}	

A classe \textbf{ServerMeta} é o núcleo do serviço de metadados, possuindo como superclasse o \textit{DefaultSingleRecoverable}, uma classe que pertence na biblioteca \textit{BFT-SMaRt}.
Dentro do construtor ela inicializa o árvore de diretórios e a lista de servidores de armazenamento, como mostra no Código~\ref{code:serv_meta_con}.

\begin{lstlisting}[basicstyle=\ttfamily\footnotesize, frame=single, caption=Declaração e construtor da classe ServerMeta, label=code:serv_meta_con]

public class ServerMeta extends DefaultSingleRecoverable {
			...
		
	public ServerMeta(int id){
		new ServiceReplica(id, this, this);
		dt   = new DirectoryTree();
		list = new ServerList(); 
	}

			...
\end{lstlisting}	

O método \textit{appExecuteOrdered}, sobre escrito da superclasse, mostrado no Código~\ref{code:serv_meta_app} serve para atender às requisições por operação onde a sua ordem da execução precisa ser considerada.
Primeiro recebe o tipo de operação para ser executado em variável \textit{reqType}, e depois chama o método correspondente para continuar com o processo.
As operações que podem ser processados em qualquer ordem são atendidas pelo método \textit{appExecuteUnordered}.

\begin{lstlisting}[basicstyle=\ttfamily\footnotesize, frame=single, caption=Métodos para atender às requisições, label=code:serv_meta_app ]		
	@Override
	public byte[] appExecuteOrdered(byte[] command, MessageContext msgCtx) {
				
			...
				
		byte[] resultBytes = null;
		
		try {
			ByteArrayInputStream in  = new ByteArrayInputStream(command);
			ObjectInputStream    ois = new ObjectInputStream(in);
			
			int reqType = ois.readInt();
			
			switch(reqType) {
				case RequestType.CREATEDIR:
				resultBytes = criateDir(ois);
				break;
				
				case RequestType.DELETEDIR:
				resultBytes = deleteDir(ois);
				break;
	
			...
				
		} catch (ClassNotFoundException | IOException e) {
	
			...
			
	}
			
	@Override
	public byte[] executeUnordered(byte[] command, MessageContext msgCtx) {	
			...
\end{lstlisting}	

Todos os métodos da classe \textit{ServerMeta} possuem um fluxo de execução padronizado desta forma; recebe requisição, processa a operação e retorna o resultado. 
O método \textit{open} mostrado no Código~\ref{code:serv_meta_open} apresenta o fluxo padrão, realizando a operação de abrir um arquivo.

\begin{lstlisting}[basicstyle=\ttfamily\footnotesize, frame=single, caption=Exemplo de método da classe ServerMeta, label=code:serv_meta_open]		
	private byte[] open(ObjectInputStream ois) throws ClassNotFoundException, IOException {
		String   currPath = (String)ois.readObject();
		String   tgtName  = (String)ois.readObject();
		long     accTime  = ois.readLong();
		
				...
		
		Directory currDir   = dt.openDirectory(currPath, accTime);
		int       result    = -1;
		long      fileSize  = 0;
		FileDFS   target    = null;
		BlockInfoList bList = null;
		
		if(currDir == null) {
			currDir = dt.getRoot();
			result  = ResultType.FAILURE;
		} else if(!currDir.existFile(tgtName)) {
			result = ResultType.FILENOTEXISTS;
		} else {
			target = currDir.getFile(tgtName);
			if( ( target.isLokedW() ) &&
				( System.currentTimeMillis()-target.getLastAccTime() ) <= 30*1000 ) 
			{
				result = ResultType.FILELOCKED;
			} else {
				target.lockR();
				bList  = target.getBlockList();
				fileSize = target.getMetadata().size();
				result = ResultType.SUCCESS;
			}
		}
	
		ByteArrayOutputStream out = new ByteArrayOutputStream();
		ObjectOutputStream    oos = new ObjectOutputStream(out);
	
		oos.writeInt(result);
		oos.writeObject(currDir.getDirEntries());
		oos.writeObject(bList);
		oos.writeLong(fileSize);
		oos.flush();
		
				...
					
		return out.toByteArray();
	}
\end{lstlisting}


\subsection{Serviço de Armazenamento}
Nesta seção a implementação do Serviço de Armazenamento será detalhada. O serviço supracitado foi desenvolvido em quatro classes, as quais são listadas a seguir.
\\

\begin{itemize}
	\item server.data;
	\begin{itemize}
	\item MetadataModule;
	\item ServerConsole;
	\item Operation;
	\item ServerData.
	\end{itemize}
\end{itemize}

A classe \textbf{ServerConsole} serve apenas como uma interface de inicialização do serviço de armazenamento, no qual o \textit{ServerConsole} verifica os parâmetros de inicialização, caso tenha algo errado é apresentado uma mensagem de erro e o programa finalizado. Em contra partida, se tudo estiver correto com os parâmetros, a classe \textit{ServerData} é instanciada.
\\

A classe \textbf{Operation} é a classe responsável por efetivamente executar as operações solicitadas ao serviço de armazenamento sobre os blocos de arquivos.
Ela herda a classe \textit{Thread} para realizar o processamento em \textit{multithread}.
Na parte de execução mostrado no Código~\ref{code:operation} faz a chamada de método adequado de acordo com tipo de operação requerida.
\begin{lstlisting}[basicstyle=\ttfamily\footnotesize, frame=single, caption=Declaração e o método de execução da classe Operation, label=code:operation]		
public class Operation extends Thread {
	
			...
			
	public void run() {
		if(verbose)
			System.out.println("Cliente conectado do IP "
			+clientSocket.getInetAddress().getHostAddress());
		
		try {
			InputStream in = clientSocket.getInputStream();
			ObjectInputStream ois = new ObjectInputStream(in);
			
			int   reqType = ois.readInt();
			Block block   = (Block)ois.readObject();
			
			switch(reqType) {
				case(RequestType.CREATE):
					create(block);
					break;
			
				case(RequestType.DELETE):
			
			...
\end{lstlisting}



\textbf{ServerData} é a classe responsável por lidar diretamente com os clientes, de modo que é ela quem sabe a porta onde o servidor deve aguardar pela conexão do cliente, a sua capacidade total de armazenamento e quanto do espaço já foi ocupado.

O trecho do código mostrado no Código~\ref{code:serv_data} é a parte principal de execução da classe \textit{ServerData}.
O fluxo começa esperando cliente,  segue para criação da instancia de \textit{Operation} e dá início à instancia criada.
\begin{lstlisting}[basicstyle=\ttfamily\footnotesize, frame=single, caption=Declaração e o método de execução da classe Operation, label=code:serv_data]		
	while(true) {
		iterations++;
		if(verbose)
			System.out.println("Aguardando cliente...");
		try {
			Socket clientSocket = serverSocket.accept();
			
			Operation op = new Operation(clientSocket, dirName, verbose);
			op.start();
		} catch (SocketException e) {
			e.printStackTrace();
			System.exit(0);
		}
	}
	
			...
\end{lstlisting}

A classe \textbf{MetaDataModule} realiza a comunicação com os servidores de metadados. Tal comunicação é realizada utilizando as facilidades do \textit{BFT-SMaRt}, no qual o \textit{MetaDataModule} age no papel de um cliente solicitando algum serviço do servidor de metadados.
\\

\subsection{Cliente}
Nesta seção a implementação do Cliente será detalhada. O serviço supracitado foi desenvolvido em cinco classes contidos no pacote \textit{client}, as quais são listadas a seguir.
\\

\begin{itemize}
	\item client
	\begin{itemize}
	\item ClientConsole;
	\item ClientDFS;
	\item ClientServerSocket;
	\item Option;
	\item ClientTest.
	\end{itemize}
\end{itemize}

Observe que a classe \textbf{ClientTest} é utilizada apenas para realizar os testes, na execução normal ela é ignorada pelo sistema e não deve ser chamada. A parte de preparação é mostrado Código~\ref{code:clent_test1}, onde são criados vários \textit{threads} da classe \textit{ClientDFS} e colocado para executar o teste.

\begin{lstlisting}[basicstyle=\ttfamily\footnotesize, frame=single, caption=Preparação de teste na classe ClientTest, label=code:clent_test1]		
	long[] values = new long[numThreads];
	Client[] c = new Client[numThreads];
	
	for (int i = 0; i < numThreads; i++) {
		try {
			Thread.sleep(100);
		} catch (InterruptedException e) {
			e.printStackTrace();
		}
	
		System.out.println("Launching client " + (initId + i));
		c[i] = new ClientTest.Client(values, i, opsType, numberOfOps, interval);
	}
		
			...
\end{lstlisting}

Agora executa o teste chamando os métodos de acordo com tipo de  operação escolhido, como mostra no Código~\ref{code:clent_test2}. 
Ao terminar a execução de todas as operações, apenas a instancia de \textbf{ClientDFS} que possui número de ID igual a ID inicial imprime o resultado e termina o teste.

\begin{lstlisting}[basicstyle=\ttfamily\footnotesize, frame=single, caption=Execução de teste na classe ClientTest, label=code:clent_test2]
	for (int i = 0; i < numberOfOps; i++, req++) {
		System.out.print("Sending req " + req + "...");
		
		try {
			switch(opsType) {
				case(READ):
					last_send_instant = System.nanoTime();
					cdfs.open("test_"+id+"_"+i);
					st.store(System.nanoTime() - last_send_instant);
					break;
				
				case(WRITE):
					last_send_instant = System.nanoTime();
					cdfs.create("test", "test_"+id+"_"+i);
					st.store(System.nanoTime() - last_send_instant);
					break;
				
			...
				
		if (id == initId) {
			System.out.println(this.id + " // Average time for " + numberOfOps + " executions (-10%) = " + st.getAverage(true) / 1000 + " us ");
			
			...
				
		}	
			
			...
\end{lstlisting}
 


A classe \textbf{ClientConsole} serve apenas como uma interface entre o usuário e o \textit{software}, no qual o \textit{ClientConsole} apresenta um menu baseado em linha de comando, no qual o usuário informa qual operação deseja executar, o \textit{ClientConsole} valida os dados fornecidos e avisa o módulo responsável por executar a operação desejada.
\\

No Código~\ref{code:clent_con} é mostrado o método para executar a operação de criação de arquivo.
Basicamente todos os métodos que executa a operação possuem esta forma; interação com usuário, chamada de método do \textit{ClientDFS} e processamento de resultado.
\begin{lstlisting}[basicstyle=\ttfamily\footnotesize, frame=single, caption=Exemplo de método da classe ClientConsole, label=code:clent_con]
		private void create(Console con) throws ClassNotFoundException, IOException  {
			System.out.println();
			System.out.println("Criar arquivo");
			String srcName  = con.readLine("Nome ou local do arquivo:\n>");
			
			if(srcName.isEmpty())
				return;
			
			int result = c.create(srcName, null);
			
			if(result == ResultType.SUCCESS)
				System.out.println("Arquivo criado");
			else
				reportError(result);
		}
\end{lstlisting}

A classe \textbf{Option} é apenas uma enumeração contendo os tipos de operações suportados pelo sistema, as quais foram apresentadas e discutidas previamente nesse trabalho.
\\

\textbf{ClientServerSocket} possibilita e gerencia as conexões \textit{Socket} realizadas entre o cliente e os servidores de dados, tal conexão são necessárias para a correta execução de operações entre o cliente e os servidores de armazenamento. As conexões entre o cliente e servidor de metadados ocorrem graças as facilidades do \textit{BFT-SMaRt}.
O método \textit{open} mostrado no Código~\ref{code:clent_socket1} serve para receber o bloco de arquivo, enviado pelo servidor de armazenamento.
Primeiramente envia o tipo de requisição junto à informação sobre bloco de arquivo desejado e logo a seguir espera o envio de dado pelo servidor.
Para tolerar a falha da rede externa, quando detecta exceção de conexão espera por 10 segundos e inicia o processo novamente, fazendo três tentativas.

\begin{lstlisting}[basicstyle=\ttfamily\footnotesize, frame=single, caption=Exemplo de método da classe ClientServerSocket, label=code:clent_socket1]	
	public byte[] open(Block block) throws ConnectException  {
		int triedCount = 0;
		
		while(true) {
			try {
				clientSocket = new Socket(blockInfo.getHostName(), blockInfo.getPort());
				System.out.println("O cliente se conectou ao servidor na porta " 
				+ blockInfo.getPort());
				
				ByteArrayOutputStream bos = new ByteArrayOutputStream();
				ObjectOutputStream    oos = new ObjectOutputStream(bos);
				
				oos.writeInt(RequestType.OPEN);
				oos.writeObject(block);
				oos.flush();
				
				OutputStream out = clientSocket.getOutputStream();
				
				out.write(bos.toByteArray());
			
				InputStream         is = clientSocket.getInputStream();
				BufferedInputStream in = new BufferedInputStream(is);
				
				while(is.available() == 0);
				
				byte[] buffer = new byte[BUFFER_SIZE];
				int length = in.read(buffer);
				System.out.println(length);
			
				return Arrays.copyOfRange(buffer, 0, length);
			} catch(ConnectException | UnknownHostException e) {
				try {
					Thread.sleep(10*1000);
				} catch (InterruptedException e1) {
					e1.printStackTrace();
					System.exit(-1);
				}
				
				triedCount++;
				if(triedCount>3) {
					System.out.println("O cliente nao conseguiu conectar no servidor");
					throw new ConnectException();
				}
			} catch (IOException e) {
				...
			}
		}
	}
\end{lstlisting}

O método \textit{send} foi implementado como privado para ser chamado pelo outros métodos públicos, o \textit{create} e o \textit{delete}, que servem para fazer o envido do bloco de arquivo e a requisição de deletar um bloco, respectivamente.
Utiliza o mesmo esquema de tolerar a falha da rede externa que foi apresentado no método acima.
A implementação é mostra do no Código~\ref{code:clent_socket2}
\begin{lstlisting}[basicstyle=\ttfamily\footnotesize, frame=single, caption=Exemplo de métodos da classe ClientServerSocket, label=code:clent_socket2]	
	public void create(Block block) throws ConnectException  {
		send(RequestType.CREATE, block);
	}
	
	public void delete(Block block) throws ConnectException  {
		send(RequestType.DELETE, block);
	}
	
				...
	
	private void send(int ReqType, Block block) throws ConnectException  {
		int triedCount = 0;
		
		while(true) {
			try {
				clientSocket = new Socket(blockInfo.getHostName(), blockInfo.getPort());
				System.out.println("O cliente se conectou ao servidor na porta " 
				+ blockInfo.getPort());
				
				ByteArrayOutputStream bos = new ByteArrayOutputStream();
				ObjectOutputStream    oos = new ObjectOutputStream(bos);
				
				oos.writeInt(ReqType);
				oos.writeObject(block);
				oos.flush();
				
				OutputStream out = clientSocket.getOutputStream();
				
				out.write(bos.toByteArray());
				
				clientSocket.close();
				
				return;
			} catch(ConnectException | UnknownHostException e) {
			
					...
					
			}
		}
	}
\end{lstlisting}

A classe \textbf{ClientDFS} pode ser considerada como a classe principal do lado cliente do sistema, visto que é nela onde estão concentrados os métodos para realização de todas as operações listadas em \textit{Option}.
Quando o método não envolve com uma operação sobre arquivo, o cliente faz a comunicação somente entre servidores de metadados.
O Código~\ref{code:clent_dfs1} mostra o método que implementa a operação de criar novo diretório, em que a requisição desta operação é enviado para servidores usando o \textit{BFT-SMaRt}, através do método \textit{invokeOrdered}.
O \textit{invokeOrdered} é chamado para requisitar as operações que a ordem de execução precisa ser considerado. Caso contrário, é chamado  \textit{invokeUnordered}.
Os métodos desta classe também possuem uma forma padrão; prepara o dado, envia para servidores, aguarda a resposta, e faz processamento restante de acordo com a resposta obtida.  
\begin{lstlisting}[basicstyle=\ttfamily\footnotesize, frame=single, caption=Exemplo de método da classe ClientDFS, label=code:clent_dfs1]	
	public int criateDir(String tgtName) throws ClassNotFoundException, IOException {
		Metadata metadata = new Metadata(System.currentTimeMillis());
		
		ByteArrayOutputStream out = new ByteArrayOutputStream();
		ObjectOutputStream    oos = new ObjectOutputStream(out);
		
		oos.writeInt(RequestType.CREATEDIR);
		oos.writeObject(currPath.toString());
		oos.writeObject(tgtName);
		oos.writeObject(metadata);
		oos.writeLong(System.currentTimeMillis());
		oos.flush();
		
		byte[]  bytes   = this.proxy.invokeOrdered(out.toByteArray());
		
		ByteArrayInputStream in  = new ByteArrayInputStream(bytes);
		ObjectInputStream    ois = new ObjectInputStream(in);
		
		int result = ois.readInt();
		currDir    = (DirEntries)ois.readObject();
		
		if(result == ResultType.FAILURE) {
			currPath = currDir.getPath();
		}
		
		return result;
	}
\end{lstlisting}
Quado o método envolve uma operação sobre arquivo, é necessário fazer comunicação com ambos tipos de servidores, o de metadados e o de armazenamento, como no método \textit{create} mostrado abaixo no Código~\ref{code:clent_dfs2}.
A comunicação com servidor de metadado possui mesma característica que foi apresentado no método acima.
No caso de comunicação com servidores de armazenamento ocorre através da classe \textit{ClientServerSocket} de forma \textit{multithread}.
Nas operações de leitura e escrita de arquivo a rotina de execução subdivide de acordo com tipo de RAID, mas o fluxo segue de forma parecida, como ocorre em alguns métodos apresentados anteriormente.
\begin{lstlisting}[basicstyle=\ttfamily\footnotesize, frame=single, caption=Exemplo de método da classe ClientDFS para operação sobre arquivos, label=code:clent_dfs2]	
	public int create(String fileName, String tgtName) throws ClassNotFoundException, IOException  {
		int result = ResultType.FAILURE;
		try {
				...
		
			byte[]  bytes   = this.proxy.invokeOrdered(out.toByteArray());
			
				...
					
			ClientServerSocket[] css = new ClientServerSocket[nServers];
			byte [] buffer = new byte[blockSize];
			
			switch(raidType) {
				case(RaidType.RAID0):
				for(int i=0; i<nServers; i++) {
					Arrays.fill(buffer, (byte) 0);
					
					bInfo = bList.get(i);
					
					bis.read(buffer, 0, blockSize);
					
					Block block = new Block(bInfo.getID(),buffer);
					
					
					css[i] = new ClientServerSocket(bInfo, block, Option.CREATE, verbose);
					css[i].start();   
				}
				for(int i=0; i<nServers; i++) {
					css[i].join();
				}
				for(int i=0; i<nServers; i++) {
					if(css[i].failure()) {
						failure(bList.get(i));
						result = ResultType.FAILURE;
						break;
					}
				}
				break;
				
				case(RaidType.RAID1):
						
			...
		
				case(RaidType.RAID5):
				
			...
			
\end{lstlisting}


\subsection{Executando o Sistema}
O código-fonte de todo o projeto pode ser encontrado no repositório do \textit{GitHub}, no seguinte endereço:  \href{https://github.com/diogoAF/tccRAID}{https://github.com/diogoAF/tccRAID}. Vale ressaltar que todas as bibliotecas necessárias, incluindo o  \textit{BFT-SMaRt}, já estão incluídas no projeto. Para a execução do sistema é necessário ao menos oito máquinas remotas, sendo três máquinas para o serviço de metadados, quatro para o serviço de armazenamento (no caso do RAID 0 ou 1, devido a sua natureza, é possível ser utilizado com apenas duas máquinas), e ao menos uma máquina executando o Cliente.
\\

O primeiro passo é criar o arquivo \textbf{host.config} dentro de um diretório chamado \textbf{config}. Localizado no diretório \textbf{bin} existe um modelo de construção deste arquivo. Ele é necessário pois é utilizado pelo \textit{BFT-SMaRt} para determinar o IP e porta de cada réplica. Ao final do arquivo deve-se inserir uma nova linha contendo o ID do servidor, endereço IP e porta pela qual ele vai escutar. Por exemplo, 7001 10.1.1.9 11100.
\\

O segundo passo é inicializar o serviço de metadados, para tal deve-se chamar a classe \textit{server.meta.ServerConsole}, a qual deve receber quatro parâmetros os quais serão listado a seguir. 

\begin{itemize}
	\item O identificador único da réplica;
	\item O tipo do RAID que será utilizado;
	\item O número de servidores de armazenamento;
	\item "v" ~caso deseje mostrar na tela as informações da execução.
	\item "n" ~caso deseje não executar em modo de teste, um modo que grava os dados em um único arquivo do disco.
\end{itemize}

Porém, antes de executar o comando de inicialização, recomenda-se criar um \textit{script} que inicialize o \textit{BFT-SMaRt} e as outras bibliotecas, para tal, basta seguir o formato apresentado logo abaixo, para conveniência, doravante vamos supor que o \textit{script} foi criado e chama-se \textit{scriptBftSmart.sh}.

\begin{lstlisting}
java -cp .:lib/BFT-SMaRt.jar:lib/slf4j-api-1.5.8.jar:lib/slf4j-jdk14-1.5.8.jar:
lib/netty-3.1.1.GA.jar:lib/commons-codec-1.5.jar $1 $2 $3 $4 $5 $6 $7 $8 $9
\end{lstlisting}

Com o \textit{script} criado, basta executar o seguinte comando (cada um em cada máquina), repare que cada máquina irá receber um identificador único, entretanto, o restante dos parâmetros são inalterados. Repare que é na inicialização do serviço de metadados que o tipo de RAID é determinado e ele não pode ser alterado em tempo de execução. Neste exemplo o serviço está sendo iniciado para tratar o RAID 0 com quatro servidores de armazenamento em modo normal e mostrando as informações da execução.
\\

\begin{lstlisting}
sh scriptBftSmart.sh server.meta.ServerConsole 0 0 4 vn
sh scriptBftSmart.sh server.meta.ServerConsole 1 0 4 vn
sh scriptBftSmart.sh server.meta.ServerConsole 2 0 4 vn
\end{lstlisting}

Com os servidores de metadados devidamente inicializados, deve iniciar o serviço de armazenamento. A classe que deve ser invocada é a \textit{server.data.ServerConsole}, a qual deve receber dois parâmetros, os quais serão listado a seguir. 

\begin{itemize}
	\item O identificador único da réplica;
	\item "v" ~caso deseje mostrar na tela as informações da execução.
\end{itemize}

No exemplo a seguir, o serviço será instanciado com quatro servidores mostrando as informações de execução na tela.
\\

\begin{lstlisting}
sh scriptBftSmart.sh server.data.ServerConsole 1001 v
sh scriptBftSmart.sh server.data.ServerConsole 1002 v
sh scriptBftSmart.sh server.data.ServerConsole 1003 v
sh scriptBftSmart.sh server.data.ServerConsole 1004 v
\end{lstlisting}

Nesse ponto os servidores de dados já deve ter estabelecido conexão com o serviço de metadados. Desta forma, só o que falta é ativar o cliente. Para tal, é possível ativá-lo de dois modos, o real e o de testes. No modo real será apresentado uma interface de terminal onde o usuário poderá interagir informando qual operação ele deseja que o sistema execute. Enquanto que no modo de teste é passado na linha de comando qual operação deve ser executada (\textit{r} para leitura ou \textit{w} para escrita), o tamanho do arquivo de teste (1,100,1000 ou 10000 todos em \textit{kilobytes}), a quantidade de \textit{threads} clientes que serão instanciadas e a quantidade de operações. Os arquivos que são utilizados no modo de teste também podem ser encontrados na pasta \textbf{bin}.
\\


Para executar um cliente em modo real, é necessário chamar a classe \textit{client.ClientConsole}, a qual recebe apenas o identificador do cliente como parâmetro. No exemplo a seguir, é inicializado um cliente de id 7001.
\\

\begin{lstlisting}
sh scriptBftSmart.sh client.ClientConsole 7001
\end{lstlisting}

No caso do cliente em modo de teste, deve-se chamar a classe \textit{client.ClientTest}, a qual teve seus parâmetros de entrada explicados no paragrafo anterior. Desta forma, no exemplo a seguir o cliente de teste vai instanciar 10 \textit{threads} clientes onde cada uma irá executar 500 operações de escrita de arquivo de 1 KB.
\\

\begin{lstlisting}
sh scriptBftSmart.sh client.ClientTest 7001 w 1 10 500
\end{lstlisting}


\section{Conclusões do Capítulo}

Nesse capítulo foi apresentado toda a modelagem do sistema proposto, abordando a arquitetura geral e partindo para pontos mais específicos como o serviço de metadados, serviço de armazenamento e o \textit{software} do lado do cliente. Também foi apresentado em detalhes as operações que o nosso sistema de arquivos distribuídos implementa, informando quais operações são e suas funções. Em seguida a implementação em si do sistema foi abordado, contendo informações técnicas de quais classes Java foram desenvolvidas e detalhes de como o desenvolvimento ocorreu, tudo dividido em subseções para cada módulo do sistema. Por fim, foi apresentado o passo-a-passo para quem deseja conseguir o código-fonte de todo o sistema e como proceder para executá-lo no modo real ou em modo de teste.
\\

\input{chapters/capitulo6}
	\section{Experimentos}
	Neste capítulo iremos detalhar os experimentos realizados sobre o nosso sistema. Explanando como e onde os experimentos foram realizados, apresentar e detalhar os dados colhidos para enfim, analisa-los para no final do capítulo apresentar a nossa conclusão.
	\\
	
	Tais experimentos tem como objetivo avaliar o desempenho de nosso sistema sobre um ambiente de alta demanda, tanto com apenas um cliente quanto múltiplas solicitações  comitentes. Para alcançar o primeiro caso, foram realizados \textbf{testes de latência} e para o último caso, \textbf{testes de vazão}. Ambos os testes foram focados em apenas duas operações em nosso sistema, leitura e escrita, sendo cada uma repetida para arquivos de quatro distintos tamanhos, 1KB, 100KB, 1MB e 10MB. Nas próximas seções estes experimentos serão descritos em maiores detalhes, porém, antes será apresentada uma subseção descrevendo o ambiente de teste.
	\\
	
	\subsubsection{O Ambiente de Teste}
	Os testes foram todos executados utilizando-se as facilidades da plataforma \textbf{Emulab-Net}, mais informações sobre a plataforma podem ser encontradas na página oficinal pelo endereço \href{https://www.emulab.net/}{https://www.emulab.net/}. Para este trabalho, basta saber que o Emulab-Net é uma ferramenta complexa para testes de rede com mais de 900 computadores (denominados como \textit{nodes}) separados em diferentes categorias que possibilitam o desenvolvimento de experimentos sofisticados nas áreas de rede de computadores e computação distribuída.
	\\
	
	Antes de se iniciar um experimento, é necessário informar quantas e quais tipos de máquinas serão utilizados além de modelar a topologia que deve ser utilizada entre os dispositivos na rede. 
	\\
	
	Para a realização de nosso experimento foram solicitadas oito máquinas, onde três seriam utilizadas para execução do serviço de metadados, quatro para o serviço de armazenamento e a última para executar o lado do cliente. As especificações técnicas de cada uma das oito máquinas encontram-se registradas na tabela~\ref{tab:exp_vm}.
	\\
	
	\capstartfalse
	\begin{table} [htb]
		\caption{Especificações Técnicas das Máquinas}
		\centering
		\begin{tabular}{|l|l|} \hline
			\textbf{Descrição} 	& \textbf{Valor} \\ \hline
			
			Tipo				& d430\\ \hline
			Classe				& PC\\ \hline
			Sistema Operacional & Ubuntu (64bits)\\ \hline
			Disco Rígido		& 200GB \\ \hline
			Memória RAM			& 4GB \\ \hline
			Nº de \textit{Cores}& 8 \\ \hline
			Velocidade do CPU	& 2.4GHz  \\ \hline
						
		\end{tabular}
		\label{tab:exp_vm}
	\end{table}
	\capstarttrue
	

	\subsubsection{Teste de latência}
	O objetivo do teste de latência é mensurar o tempo total que o sistema leva para executar grandes quantidades seguidas de uma mesma operação. Para nosso caso, decidimos testar apenas as operações de leitura e escrita de arquivos. Cada operação foi repetida 1000 vezes de forma consecutiva utilizando-se um arquivo fixo, sendo que esse procedimento foi executado com arquivos de tamanho de 1KB, 100KB, 1MB e 10MB. Todo esse procedimento foi repetido uma única vez para cada modelo de RAID suportado pelo nosso sistema, ou seja, RAID 0, RAID 1 e RAID 5. Ao fim de cada execução desse procedimento, os dados eram coletados e armazenados.
	\\
	
	\subsubsection{Teste de vazão}	
	O objetivo do teste de vazão (ou \textit{throughput}) é mensurar quantos clientes simultâneos o nosso sistema comporta sem prejudicar muito o tempo de resposta. Para tal, ele é realizado de forma extremamente simular ao teste de latência, a única diferença é que não é apenas um único cliente solicitando as operações, mas sim vários clientes concorrentes. Como em nosso experimento possuímos apenas uma máquina para o cliente, decidimos utilizar várias \textit{threads} onde cada uma age como se fosse um cliente distinto. Ao fim de cada execução do procedimento, os dados eram coletados e armazenados.
	\\
	

	\section{Resultados}
	Esta sessão apresenta as analises quantitativas e qualitativas dos resultados obtidos na realização dos testes descritos anteriormente.
	\\
	
	\subsubsection{Teste de latência}
	Foram coletadas as média das 1000 operações, descartando-se 10\% dos valores com maior desvio. A Tabela~\ref{tab:desvio_padrao} mostra a razão entre o desvio padrão e o valor da média para os diferentes tamanhos de arquivos testados. Apenas a maior razão entre os três níveis de RAID testados estão presentas na tabela, tanto para as operações de leitura quanto de escrita. Nota-se que a variação entre as latências foi pequena, consequentemente, os valores da média simples podem ser utilizados para realizar a comparação de desempenho. 
	\\
	
	\capstartfalse
	\begin{table} [htb]
		\caption{A taxa de desvio padrão}
		\centering
		\begin{tabular}{|l|c|c|c|c|} \hline
						& 1KB	& 100KB		& 1MB		& 10MB  \\ \hline
			desvio padão/média	& 1,7\%	& 6,7\%		& 3,7\%		& 3,1\% \\ \hline
		\end{tabular}
		\label{tab:desvio_padrao}
	\end{table}
	\capstarttrue
	
	A Figura~\ref{fig:latencia_l} representa o gráfico do teste de latência sobre a operação de leitura. Como no gráfico é difícil distinguir as linhas de cada RAID pois os valores de latência são próximos entre si, observando a Tabela~\ref{tab:latencia_l} é mais fácil notar a diferença entre os valores. Neste resultado a linha apresenta formato crescente, pois conforme o tamanho dos arquivos aumentam, o tempo gasto para concluir uma operação também aumenta. Tal característica leva ao incremento da latência observada.
	\\
	 
	De acordo com Tabela~\ref{tab:latencia_l} o RAID 1 apresenta o menor valor de latência entre todos os níveis de RAID para os arquivos de tamanho 1KB ou 100KB. Contudo, para arquivos maiores o resultado inverte, o RAID 1 apresenta a maior latência. Isso pode ser explicado pelo fato dos RAID 0 e RAID 5 necessitarem da reconstrução do arquivo a partir dos bloco de dados. A reconstrução de um arquivo pequeno é mais custosa do que sua transmissão pela rede, característica essa que tende a inverter com o aumento do arquivo.
	\\
	
	Pelos fatos acima apresentados e pelos valores da Tabela~\ref{tab:latencia_l}, é possível afirmar que na operação de leitura sobre arquivos de médio e grande porte a latência para cada nível de RAID obedece a relação de RAID 0 < RAID 5 < RAID 1.
	\\
	
	\begin{figure}[h]
		\begin{tabular}{lc}
			\begin{minipage}{.50\textwidth}
				\begin{center}
					
					\includegraphics[clip,width=8.0cm]{images/resultados/latencia_leitura.png}
					\caption{Gráfico de latência pata leitura}
					\label{fig:latencia_l}
					
				\end{center}
				
			\end{minipage}
			
			\begin{minipage}{.5\textwidth}
				\makeatletter
				\def\@captype{table}
				\makeatother
				\caption{Tabela de latência pata leitura(s)}
				\label{tab:latencia_l}
				\begin{center}
					\begin{tabular}{|c|c|c|c|c|} \hline
								& 1KB  & 100KB & 1MB   & 10MB  \\ \hline
						RAID 0	& 9.33 & 12.26 & 38.84 & 324.73\\ \hline
						RAID 1	& 9.01 & 12.19 & 41.50 & 348.15\\ \hline
						RAID 5	& 9.76 & 12.62 & 40.11 & 326.75\\ \hline
						
						
					\end{tabular}
				\end{center}
				
			\end{minipage}
		\end{tabular}
	\end{figure}
	
	Diferente dos resultados obtidos nos testes de leitura, as diferenças entre os valores coletados sobre a operação de escrita para os três níveis de RAID são nítidas. Fato comprovado pela Figura~\ref{fig:latencia_e} e a Tabela~\ref{tab:latencia_e}.
	\\
	
	De acordo com o Gráfico~\ref{fig:latencia_e} a linha de crescimento do RAID 1 é a mais intensa. A diferença entre os outros níveis de RAID tende a aumentar caso o tamanho do arquivo também aumente.
	\\
	
	Observando apenas ao Gráfico~\ref{fig:latencia_e} têm-se a errônea sensação de que o RAID 1 possui a maior latência para qualquer tamanho de arquivo. Contudo, estudando a Tabela~\ref{tab:latencia_e} percebe-se que para os arquivos de 1KB e 100KB o RAID 5 é quem apresenta a maior latência. Isto pode ser explicado pela mesma razão do quê ocorreu no teste de leitura. Para arquivos pequenos o tempo para dividi-los em blocos é maior do que o de envia-los. A partir de 100KB, aproximadamente, a latência do RAID 1 supera o valor dos outros níveis. 
	\\
	
	Arquivos maiores do que 100KB mantém a mesma relação observada nos testes de leitura, RAID 0 < RAID 5 < RAID 1. 
	\\
	
	\begin{figure}[h]
		\begin{tabular}{lc}
			\begin{minipage}{.50\textwidth}
				\begin{center}
					
					\includegraphics[clip,width=8.0cm]{images/resultados/latencia_escrita.png}
					\caption{Gráfico de latência pata escrita}
					\label{fig:latencia_e}
					
				\end{center}
				
			\end{minipage}
			
			\begin{minipage}{.5\textwidth}
				\makeatletter
				\def\@captype{table}
				\makeatother
				\caption{Tabela de latência pata escrita(s)}
				\label{tab:latencia_e}
				\begin{center}
					\begin{tabular}{|c|c|c|c|c|} \hline
								& 1KB  & 100KB & 1MB   & 10MB \\ \hline
						RAID 0	& 6.70 & 6.91 & 14.92 & 101.93\\ \hline
						RAID 1	& 6.79 & 7.53 & 23.79 & 213.94\\ \hline
						RAID 5	& 7.31 & 7.91 & 18.26 & 134.31\\ \hline
						
					\end{tabular}
					
				\end{center}
			\end{minipage}
		\end{tabular}
	\end{figure}
	
	\subsubsection{Teste de \textit{throughput}}
	Foram coletados apenas o maior valor de \textit{throughput} observado durante os testes. As linhas do gráfico possuem formato decrescente, pois o crescimento do arquivo leva ao aumento do tempo de transferência, de modo que a quantidade de operações executadas por segundo também diminuem.
	\\
	 
	
	O Gráfico~\ref{fig:throughput_l} e a Tabela~\ref{tab:throughput_l} foram obtidos ao fim dos experimentos de \textit{throughput} sobre as operações de leitura. Nota-se que o comportamento observado no teste de latência também ocorre no teste de \textit{throughput}. Pois para os arquivos de tamanhos de 1KB e 100KB, o RAID 1 apresenta os maiores valores de \textit{throughput}. Enquanto que para aquivos maiores o desempenho do RAID 1 despenca, ficando em útilmo lugar se comparado aos outros dois. A explicação para este fato pode ser a mesma do teste de latência, a relação entre os custos de reconstrução, transmissão de dados e o tamanho do arquivo.
	\\
	
	Para os arquivos maiores do que 10MB, as taxa de \textit{throughput} dos três níveis de RAID tendem a convergir para o mesmo valor, porém entre esse valor e 500KB o RAID 0 possui melhor desempenho. 
	\\
	
	\begin{figure}[h]
		\begin{tabular}{lc}
			\begin{minipage}{.50\textwidth}
				\begin{center}
					
					\includegraphics[clip,width=8.0cm]{images/resultados/throughput_leitura.png}
					\caption{Gráfico de throughput para leitura(ops/s)}
					\label{fig:throughput_l}
					
				\end{center}
				
			\end{minipage}
			
			\begin{minipage}{.5\textwidth}
				\makeatletter
				\def\@captype{table}
				\makeatother
				\caption{Tabela de throughput para leitura(ops/s)}
				\label{tab:throughput_l}
				\begin{center}
					\begin{tabular}{|c|c|c|c|c|} \hline
						& 1KB & 100KB & 1MB & 10MB \\ \hline
						
						RAID 0	& 167.14 & 151.63 & 95.08 & 10.63\\ \hline
						RAID 1	& 278.86 & 177.43 & 51.92 & 10.52\\ \hline
						RAID 5	& 152.79 & 135.30 & 63.34 & 10.56\\ \hline
						
					\end{tabular}
				\end{center}
				
			\end{minipage}
		\end{tabular}
	\end{figure}
	
	
	O último experimento feito é o teste de \textit{throughput} para escrita de arquivos. Diferentemente dos resultados obtidos pelos testes anteriores, em que o RAID 1 apresentava melhor desempenho apenas para os casos de arquivos pequenos, o Gráfico~\ref{fig:throughput_e} mostra que o RAID 0 apresenta a melhor taxa de \textit{throughput} desde que o arquivo não seja menor do que 1KB. Situação totalmente oposta dos testes anteriores, onde o RAID 0 apresentava bom desempenho apenas com arquivos pequenos, aqui ele apenas fraqueja lidando com arquivos pequenos. 
	\\
	
	\begin{figure}[h]
		\begin{tabular}{lc}
			\begin{minipage}{.50\textwidth}
				\begin{center}
					
					\includegraphics[clip,width=8.0cm]{images/resultados/throughput_escrita.png}
					\caption{Gráfico de throughput para escrita}
					\label{fig:throughput_e}
					
				\end{center}
				
			\end{minipage}
			
			\begin{minipage}{.5\textwidth}
				\makeatletter
				\def\@captype{table}
				\makeatother
				\caption{Tabela de throughput para escrita(ops/s)}
				\label{tab:throughput_e}
				\begin{center}
					\begin{tabular}{|c|c|c|c|c|} \hline
						& 1KB & 100KB & 1MB & 10MB \\ \hline
						
						RAID 0	& 1209.19 & 988.14 & 113.07 & 11.51\\ \hline
						RAID 1	& 1023.54 & 571.43 & 58.01  & 5.76 \\ \hline
						RAID 5	& 914.91  & 830.56 & 84.52  & 8.66 \\ \hline
						
						
					\end{tabular}
				\end{center}
				
			\end{minipage}
		\end{tabular}
	\end{figure}
	
	\section{Conclusões do Capítulo}
	
	Depois de obter os resultados dos testes foi possível verificar que o desempenho para leitura e escrita entre os três níveis de RAID respeita a relação de RAID 0 > RAID 5 > RAID 1 para a maioria dos casos. Deste modo é possível concluir que o RAID 5 agrega segurança e confiabilidade ao sistema de arquivos distribuídos com pouca degradação de desempenho, se comparado ao simples fracionamento de dados do RAID 0 ou espelhamento do RAId 1. 
	\\
\chapter{Conclusões e Trabalhos Futuros}
\section{Conclusão}

\begin{frame}{}
	\begin{itemize}
		\item Conclusões
		\item Trabalhos Futuros
	\end{itemize}
\end{frame}

\postextual

\bibliographystyle{plain}

\bibliography{bibliografia}


\end{document}


