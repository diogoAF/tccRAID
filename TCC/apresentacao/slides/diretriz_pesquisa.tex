\section{Diretriz da pesquisa}

\begin{frame}{Objetivo e Hipótese}
	
	\begin{block}{Objetivo}
		Construir um sistema de arquivos distribuídos que seja tolerante a falhas, sem prejuízo de desempenho e que consuma menor quantidade de \textit{overhead} do que o sistema apresentado pelo método de replicação de dados tradicional.
		
	\end{block}
	
	\begin{block}{Hipótese}
		Os conceitos de RAID podem ser aplicados no sistema de arquivos distribuídos afim de suprir as suas necessidades consumindo menos espaço que a replicação tradicional, visto que a quantidade redundante de dados será menor. 
		
	\end{block}
	
\end{frame}

\begin{frame}{Metodologia}
	
	\begin{block}{Estudos bibliográficos}
		\begin{itemize}
			\item Conceitos de proteção de dados da tecnologia RAID.
			\item Arquitetura do sistema de arquivos distribuídos.
		\end{itemize}
	\end{block}
	
	\begin{block}{Elaboração e desenvolvimento do sistema}
		
	\end{block}
	
	\begin{block}{Testes e coleta de dados}
		
	\end{block}
	
	\begin{block}{Análise de dados}
		\begin{itemize}
			\item Comparação entre os métodos de replicação tradicional e os diferentes tipos de RAID.
		\end{itemize}
		
	\end{block}
	
\end{frame}