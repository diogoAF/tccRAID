\section{Introdução}

%\section{Problema}
\begin{frame}{Problema}
	No contexto dos sistemas distribuídos o conceito de paralelismo é muito presente, o que acarreta no aumento de máquinas conectadas e consequentemente a probabilidade de alguma dessas máquinas sofrer alguma falha de \textit{hardware} ou \textit{software}. Quando essa falha ocorrer, é necessário que contramedidas tenham sido implementadas a fim de minimizar os danos. Uma dessas medidas é que o sistema de arquivos distribuido seja tolerante a falhas, tolerância essa que geralmente é alcançada utilizando-se a replicação total dos dados, entretanto esse \textit{overhead} ocupa espaços de armazenamento que poderiam ser utilizados para novos dados. O aumento de \textit{overhead} acarreta no incremento das despesas do sistema.
\end{frame}